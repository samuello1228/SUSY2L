VV & $3.1\pm0.6$ & \\
\hline
V$+\gamma$ & $0.0\pm0.0$ & \\
\hline
fake lepton & $1.9\pm0.4$ & \\
\hline
Total BG & $5.0\pm0.7$ & \\
\hline
Signal (400, 380) & $0.5\pm0.2$ &$-0.061$\\
\hline
Signal (400, 350) & $0.3\pm0.2$ &$-0.135$\\
\hline
Signal (400, 300) & $0.0\pm0.0$ &$-0.235$\\
\hline
Signal (400, 200) & $0.0\pm0.0$ &$-0.235$\\
\hline
Signal (400, 100) & $0.0\pm0.0$ &$-0.235$\\
\hline
