\documentclass[mathserif,serif]{beamer}
\usepackage{tabularx}
\setbeamertemplate{footline}[frame number]
% \useoutertheme{infolines}
\usepackage{slidesphysics}
\graphicspath{{../plot/}}

\title[]{Signal Optimization}
\author[]
{
Samuel Lo \inst{1}
\and
Yanjun Tu  \inst{1}
\and
Dongliang Zhang  \inst{2}
}
\institute[]
{
\inst{1}
The University of Hong Kong
\and
\inst{2}
University of Michigan
}
\date[]{\today}

\newcommand\Wider[2][3em]{%
\makebox[\linewidth][c]{%
\begin{minipage}{\dimexpr\textwidth+#1\relax}
\raggedright
\centering#2
\end{minipage}%
}%
}

\begin{document}
\frame{\titlepage}

\begin{frame}{Introduction}
\begin{itemize}
\item In this week, I am so confused that what I should do.
\begin{itemize}
\item Wednesday: Jeanette send me the to-do-list: Dani and I should agree on SR optimization (compare and agree on the final signal selection cuts), and Dongliang think it is good.
\item Thursday to Fridays: try to talk with Dani, and follow Dani's suggestion, just cross-check the yields from Dani's cut, and wait for her result.
\item Weekend: Dongliang think that I need to develop my tool to do my own optimization, cannot wait for Dani's result. I follow Dongliang's suggestion, and start to develop my code.
\item Monday: Dani has new results, but I need to find my optimized cut first, and want to compare my optimized cut with her.
\item Tuesday: Get some result for my own optimization, Jeanette knows I am doing my own optimization. Jeanette think I should not do my own optimization, but Dongliang do not think so.
\end{itemize}
\item Used the updated cross section for signal.
\item See separate slides for N-1 plots
\end{itemize}
\end{frame}

\section{optimization}
\begin{frame}{Method to deal with low statistics}
\begin{itemize}
\item If the weighted yield for a type of BG is negative, set the weighted yield to 0.
\item Weighted yield for signal need to be larger than 1.
\item nSig/nSigError $>$ 2.
\item If unweighted yield for total BG is 0, set the weighted yield to 1.
\item For METRel, pt1 and pt2, there is no upper cut if gain in significance $<$ 20\%.
\item For mlj/mljj, there is no lower cut.
\end{itemize}
\end{frame}

\begin{frame}{significance table}
\begin{itemize}
\item The optimized cuts for (175,0) signal point can be used.
\item Combined significance: \\
(175,0): 11.9, (165,35): 4.2, (400,0): 2.9
\end{itemize}
\includegraphics[width=\textwidth]{data/optimization/dongliang.png}
\end{frame}

\begin{frame}{optimization}
\tiny
SR\_SS\_ee\_1\_opt: (175, 0): \\
$\pt^{l1}$ $\geq 35$ \\
$\pt^{l2}$ $\geq 30$ \\
$\pt^{ll}$ $\geq 50$ \\
$m_{T2}$ $\geq 65$ \\
$E_{\text{T}}^{\text{miss,rel}}$ $\geq 55$ \\
$m_{\text{eff}}$ $\geq 230$ \\
$m_{\text{T}}^{\text{max}}$ $\geq 75$ \\
$m_{lj}$/$m_{ljj}$ $<90$ \\

\begin{tabular}{|c|c|c|}
\hline
& Number of events & Significance \\
\hline
Z+jets & $0.000000\pm0.000000$ (0) & \\
\hline
W+jets & $0.008016\pm0.008016$ (1) & \\
\hline
$t\bar{t}$ & $0.220227\pm0.220227$ (1) & \\
\hline
single top & $0.000000\pm0.000000$ (0) & \\
\hline
$t\bar{t}+V$ & $0.030249\pm0.011923$ (11) & \\
\hline
multi top & $0.000000\pm0.000000$ (0) & \\
\hline
VV & $-0.414568\pm1.055392$ (125) & \\
\hline
V$+\gamma$ & $0.052589\pm0.037224$ (2) & \\
\hline
VVV & $0.045041\pm0.017974$ (10) & \\
\hline
Higgs & $0.001162\pm0.002303$ (3) & \\
\hline
Total BG & $0.357284\pm0.224545$ (153) & \\
\hline
Signal (175, 0) & $4.60\pm1.09$ (19) &$3.69$\\
\hline
Signal (165, 35) & $2.27\pm0.92$ (7) &$2.09$\\
\hline
Signal (400, 0) & $1.22\pm0.13$ (106) &$1.13$\\
\hline

\end{tabular}
\end{frame}

\begin{frame}{optimization}
\tiny
SR\_SS\_mumu\_1\_opt: (162.5,12.5): \\
$\pt^{l1}$ $\geq 50$ \\
$\pt^{l2}$ $\geq 20$ \\
$\pt^{ll}$ $\geq 20$ \\
$m_{T2}$ $\geq 15$ \\
$E_{\text{T}}^{\text{miss,rel}}$ $\geq 25$ \\
$m_{\text{eff}}$ $\geq 200$ \\
$m_{\text{T}}^{\text{max}}$ $\geq 105$ \\

\begin{tabular}{|c|c|c|}
\hline
& Number of events & Significance \\
\hline
VV & $6.900575\pm0.773596$ (1230) & \\
\hline
Higgs & $1.074043\pm0.619368$ (4) & \\
\hline
W+jets & $0.306715\pm0.306715$ (1) & \\
\hline
VVV & $0.302947\pm0.055987$ (45) & \\
\hline
$t\bar{t}$ & $0.268577\pm0.268577$ (1) & \\
\hline
single top & $0.235893\pm0.235893$ (1) & \\
\hline
$t\bar{t}+V$ & $0.091507\pm0.024986$ (26) & \\
\hline
V$+\gamma$ & $0.000000\pm0.000000$ (0) & \\
\hline
multi top & $0.000000\pm0.000000$ (0) & \\
\hline
Z+jets & $0.000000\pm0.000000$ (0) & \\
\hline
Total BG & $9.180256\pm1.098944$ (1308) & \\
\hline
(162.5,12.5) & $4.66\pm0.82$ (42) &$0.88$\\
\hline
(175.0,0.0) & $4.57\pm0.84$ (34) &$0.86$\\
\hline
(450.0,0.0) & $0.15\pm0.08$ (4) &$-0.16$\\
\hline

\end{tabular}
\end{frame}

\begin{frame}{optimization}
\tiny
SR\_SS\_emu\_1\_opt: (175, 0): \\
$\pt^{l1}$ $\geq 45$ \\
$\pt^{l2}$ $\geq 25$ \\
$\pt^{ll}$ $\geq 70$ \\
$m_{T2}$ $\geq 70$ \\
$|\Delta\eta_{ll}|$ $<1.5$ \\
$E_{\text{T}}^{\text{miss,rel}}$ $\geq 100$ \\
$m_{\text{eff}}$ $\geq 200$ \\
$m_{\text{T}}^{\text{max}}$ $\geq 70$ \\
$m_{lj}$/$m_{ljj}$ $<90$ \\

\begin{tabular}{|c|c|c|}
\hline
& Number of events & Significance \\
\hline
Z+jets & $0.015010\pm0.015010$ (1) & \\
\hline
W+jets & $0.007757\pm0.007757$ (1) & \\
\hline
$t\bar{t}$ & $0.273398\pm0.273398$ (1) & \\
\hline
single top & $0.000000\pm0.000000$ (0) & \\
\hline
$t\bar{t}+V$ & $0.005722\pm0.016653$ (11) & \\
\hline
multi top & $0.000000\pm0.000000$ (0) & \\
\hline
VV & $1.228071\pm0.320612$ (185) & \\
\hline
V$+\gamma$ & $0.425192\pm0.294809$ (3) & \\
\hline
VVV & $0.109876\pm0.030820$ (16) & \\
\hline
Higgs & $0.006179\pm0.003624$ (3) & \\
\hline
Total BG & $2.071205\pm0.515729$ (221) & \\
\hline
Signal (175, 0) & $7.59\pm2.34$ (25) &$2.95$\\
\hline
Signal (165, 35) & $6.36\pm1.37$ (23) &$2.54$\\
\hline
Signal (400, 0) & $1.75\pm0.16$ (145) &$0.68$\\
\hline

\end{tabular}
\end{frame}

\begin{frame}{optimization}
\tiny
SR\_SS\_ee\_2\_opt: (175, 0): \\
$\pt^{l1}$ $\geq 35$ \\
$\pt^{l2}$ $\geq 25$ \\
$\pt^{ll}$ $\geq 40$ \\
$m_{T2}$ $\geq 60$ \\
$E_{\text{T}}^{\text{miss,rel}}$ $\geq 20$ \\
$m_{\text{eff}}$ $\geq 100$ \\
$m_{\text{T}}^{\text{max}}$ $\geq 110$ \\
$m_{lj}$/$m_{ljj}$ $<120$ \\

\begin{tabular}{|c|c|c|}
\hline
& Number of events & Significance \\
\hline
Z+jets & $0.116844\pm0.116844$ (1) & \\
\hline
W+jets & $0.177670\pm0.245104$ (2) & \\
\hline
$t\bar{t}$ & $0.000000\pm0.000000$ (0) & \\
\hline
single top & $0.000000\pm0.000000$ (0) & \\
\hline
$t\bar{t}+V$ & $0.046698\pm0.016930$ (21) & \\
\hline
multi top & $0.000000\pm0.000000$ (0) & \\
\hline
VV & $1.136181\pm0.261530$ (320) & \\
\hline
V$+\gamma$ & $0.773702\pm0.458459$ (4) & \\
\hline
VVV & $0.055400\pm0.020261$ (10) & \\
\hline
Higgs & $0.527745\pm0.523065$ (7) & \\
\hline
Total BG & $2.834240\pm0.791584$ (365) & \\
\hline
Signal (175, 0) & $6.07\pm1.48$ (22) &$2.04$\\
\hline
Signal (165, 35) & $4.47\pm1.13$ (16) &$1.53$\\
\hline
Signal (400, 0) & $0.66\pm0.11$ (49) &$0.03$\\
\hline

\end{tabular}
\end{frame}

\begin{frame}{optimization}
\tiny
SR\_SS\_mumu\_2\_opt: (162.5,12.5): \\
$\pt^{l1}$ $\geq 25$ \\
$\pt^{l2}$ $\geq 30$ \\
$\pt^{ll}$ $\geq 30$ \\
$m_{T2}$ $\geq 20$ \\
$|\Delta\eta_{ll}|$ $<1.5$ \\
$E_{\text{T}}^{\text{miss,rel}}$ $\geq 35$ \\
$m_{\text{eff}}$ $\geq 220$ \\
$m_{\text{T}}^{\text{max}}$ $\geq 80$ \\
$m_{lj}$/$m_{ljj}$ $<120$ \\

\begin{tabular}{|c|c|c|}
\hline
& Number of events & Significance \\
\hline
Z+jets & $0.000000\pm0.000000$ (0) & \\
\hline
W+jets & $0.000000\pm0.000000$ (0) & \\
\hline
$t\bar{t}$ & $0.000000\pm0.000000$ (0) & \\
\hline
single top & $0.000000\pm0.000000$ (0) & \\
\hline
$t\bar{t}+V$ & $0.034675\pm0.018342$ (25) & \\
\hline
multi top & $0.000000\pm0.000000$ (0) & \\
\hline
VV & $2.623277\pm0.282610$ (600) & \\
\hline
V$+\gamma$ & $0.000000\pm0.000000$ (0) & \\
\hline
VVV & $0.092981\pm0.056192$ (5) & \\
\hline
Higgs & $0.392183\pm0.393018$ (5) & \\
\hline
Total BG & $3.143116\pm0.487674$ (635) & \\
\hline
Signal (175, 0) & $11.39\pm1.93$ (41) &$3.65$\\
\hline
Signal (165, 35) & $6.82\pm1.50$ (24) &$2.38$\\
\hline
Signal (400, 0) & $0.94\pm0.11$ (78) &$0.20$\\
\hline

\end{tabular}
\end{frame}

\begin{frame}{optimization}
\tiny
SR\_SS\_emu\_2\_opt: (162.5,12.5): \\
$\pt^{l1}$ $\geq 20$ \\
$\pt^{l2}$ $\geq 20$ \\
$m_{T2}$ $\geq 85$ \\
$|\Delta\eta_{ll}|$ $<2$ \\
$E_{\text{T}}^{\text{miss,rel}}$ $\geq 90$ \\
$m_{\text{T}}^{\text{max}}$ $\geq 130$ \\

\begin{tabular}{|c|c|c|}
\hline
& Number of events & Significance \\
\hline
VV & $2.656941\pm0.689645$ (860) & \\
\hline
single top & $0.416387\pm0.416387$ (1) & \\
\hline
$t\bar{t}+V$ & $0.271615\pm0.038882$ (108) & \\
\hline
$t\bar{t}$ & $0.264376\pm0.264376$ (1) & \\
\hline
VVV & $0.250505\pm0.051110$ (36) & \\
\hline
W+jets & $0.090438\pm0.079085$ (2) & \\
\hline
V$+\gamma$ & $0.041288\pm0.041288$ (1) & \\
\hline
Higgs & $0.012972\pm0.007562$ (18) & \\
\hline
Z+jets & $0.004904\pm0.004904$ (1) & \\
\hline
multi top & $0.000000\pm0.000000$ (0) & \\
\hline
Total BG & $4.009427\pm0.855013$ (1028) & \\
\hline
(162.5,12.5) & $4.05\pm0.70$ (38) &$1.21$\\
\hline
(175.0,0.0) & $4.22\pm0.78$ (32) &$1.26$\\
\hline
(450.0,0.0) & $0.55\pm0.14$ (17) &$-0.03$\\
\hline

\end{tabular}
\end{frame}



\section{Conclusion}
\begin{frame}{Conclusion}
\begin{itemize}
\item Conclusion:
\begin{itemize}
\item We should see significant excess if the signal exist
\item We can exclude the signal if they are not exist
\end{itemize}
\item Few issues need to further study:
\begin{itemize}
\item The global maximum for significance is difficult to find.
\item The low statistics of background sample.
\item need to understand some of the optimization results.
\end{itemize}
\end{itemize}
\end{frame}

\section{Plan}
\begin{frame}{Plan}
\begin{itemize}
\item Dani and Samuel will agree on the SR selection (compare and agree on the final signal selection cuts).
\item Dongliang and Samuel will work on data/MC plots after pre-selection level (all lepton cuts, possibly also jets and b-jets cuts).
\item Plot the electron eta to make a decision about whether we use the electrons in the crack region.
\item Estimate charge flip BG and fake BG from data, both in SR and pre-selection level.
\end{itemize}
\end{frame}

\begin{frame}
\begin{center}
\huge
Backup
\end{center}
\end{frame}

\begin{frame}{Selection in run1 SR}
\includegraphics[width=\textwidth]{data/photo/SRcutrun1.png} \\
\url{https://arxiv.org/pdf/1501.07110.pdf}
\end{frame}

\begin{frame}
\frametitle{significance calculation}
\begin{itemize}
\item RooStats::NumberCountingUtils::BinomialExpZ(S,B,$\delta$B)
\item $\delta$B = 0.3
\end{itemize}
\end{frame}

\begin{frame}[fragile]
\frametitle{Signal sample}
\small
Sample Name(p2972 tag):
\tiny
\begin{verbatim}
mc15_13TeV.993820.MGPy8EG_A14N13LO_C1N2_Wh_2L_175_0.merge.DAOD_SUSY2.e5678_a766_a821_r7676_p2949_p2972
mc15_13TeV.993821.MGPy8EG_A14N13LO_C1N2_Wh_2L_165_35.merge.DAOD_SUSY2.e5678_a766_a821_r7676_p2949_p2972
mc15_13TeV.993822.MGPy8EG_A14N13LO_C1N2_Wh_2L_400_0.merge.DAOD_SUSY2.e5678_a766_a821_r7676_p2949_p2972
\end{verbatim}
\end{frame}

\begin{frame}[fragile]
\frametitle{Data}
\small
use both 2015 and 2016 data (3212.96 + 32861.6) /pb
\tiny
\begin{verbatim}
GRL:
GoodRunsLists/data16_13TeV/20161101/physics_25ns_20.7.xml
GoodRunsLists/data15_13TeV/20160720/physics_25ns_20.7.xml
\end{verbatim}
\end{frame}

\begin{frame}{MC BG}
p-tag: p2949
\end{frame}

\begin{frame}[fragile]
\small
Trigger list:\\
\scriptsize
\begin{verbatim}
2015
HLT_2e12_lhloose_L12EM10VH
HLT_e17_lhloose_mu14
HLT_mu18_mu8noL1

2016(A-D3)
HLT_2e17_lhvloose_nod0
HLT_e17_lhloose_nod0_mu14
HLT_mu20_mu8noL1

2016(D3-)
HLT_2e17_lhvloose_nod0
HLT_e17_lhloose_nod0_mu14
HLT_mu22_mu8noL1
\end{verbatim}
\end{frame}

\begin{frame}{Object Definitions}
\small
Tool: AnalysisBase 2.4.31, SUSYTools-00-08-60\\

\centering
\begin{table}
\small
\begin{tabularx}{\textwidth}{p{1.5cm} | p{3cm} | p{3cm} | p{3cm}}
& \textbf{Electron} & \textbf{Muon} & \textbf{Jet}\\
\hline
\textbf{Baseline}
& - $p_T>10$ GeV \newline - $|\eta^{cluster}| < 2.47$ \newline - LooseAndBLayerLLH
& - $p_T>10$ GeV \newline - $|\eta| < 2.7$ \newline - Medium
& - $p_T>20$ GeV \\
\hline
\textbf{Signal}
& - $p_T > 25$ GeV \newline - $|\eta^{cluster}| < 2.47$ \newline - TightLLH \newline - GradientLoose \newline - $|z_0 \sin \theta| < 0.5$mm \newline - $|d_0/\sigma_{d_0}| < 5$
& - $p_T > 25$ GeV \newline - $|\eta| < 2.7$ \newline - Medium \newline - GradientLoose \newline - $|z_0 \sin \theta| < 0.5$mm \newline - $|d_0/\sigma_{d_0}| < 3$
& - $p_T > 20$ GeV \newline - $|\eta|<2.8$ \newline \newline - $|JVT| > 0.59$ \newline if $p_T < 60$ GeV \newline and $|\eta| < 2.4$
\end{tabularx}
\end{table}

\raggedright
Selection:
\begin{itemize}
%\item Trigger selection
\item Exactly 2 baseline leptons and exactly 2 signal leptons
\end{itemize}

\tiny
Note: \\
Pileup reweighting is applied. \\
Scale factor for reconstruction, isolation, ID and trigger is applied.
\end{frame}

\begin{frame}
\frametitle{Definition of jets}
\normalsize
\begin{itemize}
\item Central jets: $\pt>20$ GeV, $|\eta|<2.4$, no b-tagged
\item B-jets: b-tagged
\end{itemize}
\end{frame}

\begin{frame}
\frametitle{definition of variables}
\normalsize
\begin{itemize}
\item HT: Sum of the $p_T$ of all signal jets and the two leptons.
\item R2 = MET / (MET + pt1 + pt2)
\item l12\_dPhi: difference in phi between the two leptons.
\item l12\_MET\_dPhi: difference in phi between MET and the sum of 4-momentum of the two leptons.
\end{itemize}
\end{frame}

%\begin{frame}{Expected number of events \\ For SR\_SS\_ee\_1}
\vspace{5mm}
\begin{tabular}{|c|c|c|}
\hline
& Number of events & Significance \\
\hline
Z+jets & $18.9\pm19.8$ & \\
\hline
W+jets & $3.3\pm2.1$ & \\
\hline
top & $69.1\pm5.8$ & \\
\hline
VV & $14.1\pm2.0$ & \\
\hline
V$+\gamma$ & $12.2\pm5.5$ & \\
\hline
VVV & $0.4\pm0.1$ & \\
\hline
Total BG & $117.9\pm21.6$ & \\
\hline
Signal (175, 0) & $7.1\pm1.2$ &$-0.009$\\
\hline
Signal (165, 35) & $2.4\pm0.4$ &$-0.134$\\
\hline
Signal (400, 0) & $9.8\pm0.8$ &$0.062$\\
\hline

\end{tabular}
\end{frame}

\begin{frame}{Expected number of events \\ For SR\_SS\_mumu\_1}
\vspace{5mm}
\begin{tabular}{|c|c|c|}
\hline
& Number of events & Significance \\
\hline
VV & $6.8\pm0.7$ & \\
\hline
V$+\gamma$ & $0.0\pm0.0$ & \\
\hline
Total BG & $6.8\pm0.7$ & \\
\hline
Signal (400, 380) & $0.1\pm0.0$ &$-0.201$\\
\hline
Signal (500, 450) & $0.4\pm0.0$ &$-0.106$\\
\hline
Signal (400, 300) & $2.6\pm0.2$ &$0.522$\\
\hline
Signal (400, 200) & $2.1\pm0.2$ &$0.380$\\
\hline
Signal (400, 100) & $1.7\pm0.3$ &$0.262$\\
\hline

\end{tabular}
\end{frame}

\begin{frame}{Expected number of events \\ For SR\_SS\_emu\_1}
\vspace{5mm}
\begin{tabular}{|c|c|c|}
\hline
& Number of events & Significance \\
\hline
Z+jets & $0.1\pm0.0$ & \\
\hline
W+jets & $5.0\pm2.4$ & \\
\hline
top & $39.9\pm3.6$ & \\
\hline
VV & $14.9\pm1.8$ & \\
\hline
V$+\gamma$ & $1.5\pm0.5$ & \\
\hline
VVV & $0.4\pm0.1$ & \\
\hline
Total BG & $61.9\pm4.7$ & \\
\hline
Signal (175, 0) & $8.1\pm1.2$ &$0.191$\\
\hline
Signal (165, 35) & $5.5\pm0.6$ &$0.068$\\
\hline
Signal (400, 0) & $15.0\pm1.1$ &$0.501$\\
\hline

\end{tabular}
\end{frame}

\begin{frame}{Expected number of events \\ For SR\_SS\_ee\_2}
\vspace{5mm}
\begin{tabular}{|c|c|c|}
\hline
& Number of events & Significance \\
\hline
Z+jets & $1.9\pm2.0$ & \\
\hline
W+jets & $0.5\pm0.5$ & \\
\hline
top & $12.4\pm1.9$ & \\
\hline
VV & $6.3\pm2.6$ & \\
\hline
V$+\gamma$ & $2.2\pm1.1$ & \\
\hline
VVV & $0.1\pm0.0$ & \\
\hline
Total BG & $23.5\pm3.9$ & \\
\hline
Signal (175, 0) & $1.9\pm0.4$ &$0.020$\\
\hline
Signal (165, 35) & $0.9\pm0.2$ &$-0.103$\\
\hline
Signal (400, 0) & $4.4\pm0.6$ &$0.294$\\
\hline

\end{tabular}
\end{frame}

\begin{frame}{Expected number of events \\ For SR\_SS\_mumu\_2}
\vspace{5mm}
\begin{tabular}{|c|c|c|}
\hline
& Number of events & Significance \\
\hline
Z+jets & $89.3\pm27.3$ & \\
\hline
W+jets & $1.1\pm0.6$ & \\
\hline
top & $10.7\pm1.5$ & \\
\hline
VV & $16.3\pm0.7$ & \\
\hline
V$+\gamma$ & $2.5\pm1.3$ & \\
\hline
VVV & $0.3\pm0.1$ & \\
\hline
Higgs & $6.5\pm3.7$ & \\
\hline
Total BG & $126.6\pm27.7$ & \\
\hline
Signal (175, 0) & $25.4\pm3.4$ &$0.413$\\
\hline
Signal (165, 35) & $21.3\pm2.6$ &$0.318$\\
\hline
Signal (400, 0) & $1.4\pm0.1$ &$-0.163$\\
\hline

\end{tabular}
\end{frame}

\begin{frame}{Expected number of events \\ For SR\_SS\_emu\_2}
\vspace{5mm}
\begin{tabular}{|c|c|c|}
\hline
& Number of events & Significance \\
\hline
Z+jets & $0.2\pm0.1$ & \\
\hline
W+jets & $0.8\pm0.5$ & \\
\hline
top & $14.6\pm2.3$ & \\
\hline
VV & $6.5\pm0.7$ & \\
\hline
V$+\gamma$ & $1.1\pm0.8$ & \\
\hline
VVV & $0.2\pm0.1$ & \\
\hline
Higgs & $1.3\pm0.6$ & \\
\hline
Total BG & $24.7\pm2.6$ & \\
\hline
Signal (175, 0) & $12.8\pm1.9$ &$1.078$\\
\hline
Signal (165, 35) & $12.6\pm2.0$ &$1.064$\\
\hline
Signal (400, 0) & $1.7\pm0.2$ &$-0.009$\\
\hline

\end{tabular}
\end{frame}


%\input{data/plot_SR_SS_run1.tex}

\end{document}
