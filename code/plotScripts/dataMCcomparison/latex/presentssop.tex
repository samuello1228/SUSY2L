\documentclass[mathserif,serif]{beamer}
\usepackage{tabularx}
\setbeamertemplate{footline}[frame number]
% \useoutertheme{infolines}
\usepackage{slidesphysics}
\graphicspath{{../plot/}}

\title[]{Signal Optimization}
\author[]
{
Samuel Lo \inst{1}
\and
Yanjun Tu  \inst{1}
\and
Dongliang Zhang  \inst{2}
}
\institute[]
{
\inst{1}
The University of Hong Kong
\and
\inst{2}
University of Michigan
}
\date[]{\today}

\newcommand\Wider[2][3em]{%
\makebox[\linewidth][c]{%
\begin{minipage}{\dimexpr\textwidth+#1\relax}
\raggedright
\centering#2
\end{minipage}%
}%
}

\begin{document}
\frame{\titlepage}

\begin{frame}{Introduction}
\begin{itemize}
\item In this week, I am so confused that what I should do.
\begin{itemize}
\item Wednesday: Jeanette send me the to-do-list: Dani and I should agree on SR optimization (compare and agree on the final signal selection cuts), and Dongliang think it is good.
\item Thursday to Fridays: try to talk with Dani, and follow Dani's suggestion, just cross-check the yields from Dani's cut, and wait for her result.
\item Weekend: Dongliang think that I need to develop my tool to do my own optimization, cannot wait for Dani's result. I follow Dongliang's suggestion, and start to develop my code.
\item Monday: Dani has new results, but I need to find my optimized cut first, and want to compare my optimized cut with her.
\item Tuesday: Get some result for my own optimization, Jeanette knows I am doing my own optimization. Jeanette think I should not do my own optimization, but Dongliang do not think so.
\end{itemize}
\item Used the updated cross section for signal.
\item I do not have enough time to make slides for N-1 plots.
\end{itemize}
\end{frame}

\section{optimization}
\begin{frame}{Method to deal with low statistics}
\begin{itemize}
\item If the weighted yield for a type of BG is negative, set the weighted yield to 0.
\item Weighted yield for signal need to be larger than 1.
\item nSig/nSigError $>$ 2.
\item If unweighted yield for total BG is 0, set the weighted yield to 1.
\item For METRel, pt1 and pt2, there is no upper cut if gain in significance $<$ 20\%.
\item For mlj/mljj, there is no lower cut.
\end{itemize}
\end{frame}

\begin{frame}{significance table}
\begin{itemize}
\item The optimized cuts for (175,0) signal point can be used.
\item Combined significance: \\
(175,0): 11.9, (165,35): 4.2, (400,0): 2.9
\end{itemize}
\includegraphics[width=\textwidth]{data/optimization/dongliang.png}
\end{frame}

\begin{frame}[fragile]{optimization}
\tiny
\begin{verbatim}
SR_SS_ee_1_opt: (175, 0):
45 <= pt1 < -1
20 <= pt2 < -1
0 <= ptll < -1
70 <= mTtwo < -1
0 <= fabs(eta1-eta2) < -1
55 <= METRel < -1
280 <= meff < 430
135 <= mtm < -1
0 <= mlj < 90

Zjets: 0 +/- 0 (0)
Wjets: 0 +/- 0 (0)
top: 0 +/- 0 (0)
VV: -0.0121278 +/- 0.280146 (50)
Vgamma: 0 +/- 0 (0)
VVV: 6.26348e-05 +/- 6.26348e-05 (1)
Higgs: 0 +/- 0 (0)
Total BG: 6.26348e-05 +/- 6.26348e-05 (51)

Signal (175, 0): 1.39928 +/- 0.582325 (6), Significance: 4.71305
Signal (165, 35): 0 +/- 0 (0), Significance: -3.26392
Signal (400, 0): 0.558645 +/- 0.0989239 (46), Significance: 2.55986
\end{verbatim}
\end{frame}

\begin{frame}[fragile]{optimization}
\tiny
\begin{verbatim}
SR_SS_mumu_1_opt: (175, 0):
30 <= pt1 < -1
20 <= pt2 < -1
0 <= ptll < -1
95 <= mTtwo < 110
0 <= fabs(eta1-eta2) < 1.5
90 <= METRel < -1
200 <= meff < -1
140 <= mtm < -1
0 <= mlj < 90

Zjets: 0 +/- 0 (0)
Wjets: 0 +/- 0 (0)
top: 0 +/- 0 (0)
VV: -0.0682857 +/- 0.118165 (15)
Vgamma: 0 +/- 0 (0)
VVV: 0.000545925 +/- 0.000545925 (1)
Higgs: 0 +/- 0 (0)
Total BG: 0.000545925 +/- 0.000545925 (16)

Signal (175, 0): 1.96123 +/- 0.765857 (7), Significance: 5.01703
Signal (165, 35): 0 +/- 0 (0), Significance: -2.67333
Signal (400, 0): 0.124436 +/- 0.0389975 (11), Significance: 0.207614
\end{verbatim}
\end{frame}

\begin{frame}[fragile]{optimization}
\tiny
\begin{verbatim}
SR_SS_emu_1_opt: (175, 0):
30 <= pt1 < -1
35 <= pt2 < -1
0 <= ptll < 120
95 <= mTtwo < -1
0 <= fabs(eta1-eta2) < -1
75 <= METRel < -1
230 <= meff < -1
70 <= mtm < -1
0 <= mlj < 90

Zjets: 0 +/- 0 (0)
Wjets: 0 +/- 0 (0)
top: 0.00927442 +/- 0.00985018 (5)
VV: -0.442302 +/- 0.637069 (32)
Vgamma: 0 +/- 0 (0)
VVV: 0.018226 +/- 0.0111345 (4)
Higgs: 0 +/- 0 (0)
Total BG: 0.0275004 +/- 0.0148661 (41)

Signal (175, 0): 4.86875 +/- 2.15685 (14), Significance: 6.06079
Signal (165, 35): 1.31551 +/- 0.543934 (6), Significance: 2.43184
Signal (400, 0): 0.398847 +/- 0.0677367 (37), Significance: 0.680404
\end{verbatim}
\end{frame}

\begin{frame}[fragile]{optimization}
\tiny
\begin{verbatim}
SR_SS_ee_2_opt: (175, 0):
30 <= pt1 < -1
20 <= pt2 < -1
0 <= ptll < -1
85 <= mTtwo < 100
0 <= fabs(eta1-eta2) < -1
10 <= METRel < -1
240 <= meff < 370
110 <= mtm < -1
0 <= mlj < 120

Zjets: 0 +/- 0 (0)
Wjets: 0 +/- 0 (0)
top: 0.00365115 +/- 0.0100206 (6)
VV: -0.0432296 +/- 0.112286 (18)
Vgamma: 0 +/- 0 (0)
VVV: 0 +/- 0 (0)
Higgs: 0.00259053 +/- 0.00185347 (2)
Total BG: 0.00624168 +/- 0.0101906 (26)

Signal (175, 0): 2.03314 +/- 0.834362 (7), Significance: 4.10945
Signal (165, 35): 0 +/- 0 (0), Significance: -1.91897
Signal (400, 0): 0.0177716 +/- 0.0125684 (2), Significance: -1.27025
\end{verbatim}
\end{frame}

\begin{frame}[fragile]{optimization}
\tiny
\begin{verbatim}
SR_SS_mumu_2_opt: (175, 0):
50 <= pt1 < -1
30 <= pt2 < -1
60 <= ptll < -1
70 <= mTtwo < -1
0 <= fabs(eta1-eta2) < 1.5
0 <= METRel < -1
200 <= meff < -1
75 <= mtm < -1
0 <= mlj < 120

Zjets: 0 +/- 0 (0)
Wjets: 0 +/- 0 (0)
top: 0.0155577 +/- 0.0116699 (11)
VV: 0.472923 +/- 0.116725 (126)
Vgamma: 0 +/- 0 (0)
VVV: 0.0932022 +/- 0.0559424 (5)
Higgs: 0.00298778 +/- 0.00454967 (6)
Total BG: 0.58467 +/- 0.130043 (148)

Signal (175, 0): 6.47802 +/- 1.50372 (22), Significance: 4.18942
Signal (165, 35): 2.57524 +/- 0.822315 (10), Significance: 1.9832
Signal (400, 0): 0.720309 +/- 0.102326 (59), Significance: 0.43552
\end{verbatim}
\end{frame}

\begin{frame}[fragile]{optimization}
\tiny
\begin{verbatim}
SR_SS_emu_2_opt: (175, 0):
30 <= pt1 < -1
25 <= pt2 < -1
0 <= ptll < -1
65 <= mTtwo < -1
0 <= fabs(eta1-eta2) < 2
85 <= METRel < -1
200 <= meff < -1
160 <= mtm < 215
0 <= mlj < 120

Zjets: 0 +/- 0 (0)
Wjets: -0.0546401 +/- 0.0546401 (1)
top: 0.0091173 +/- 0.00862592 (8)
VV: -0.0114641 +/- 0.0777707 (32)
Vgamma: 0 +/- 0 (0)
VVV: 0 +/- 0 (0)
Higgs: 0.00476373 +/- 0.00333049 (5)
Total BG: 0.013881 +/- 0.00924655 (46)

Signal (175, 0): 2.94275 +/- 0.842809 (13), Significance: 4.77428
Signal (165, 35): 1.40231 +/- 0.575725 (6), Significance: 2.85739
Signal (400, 0): 0.477849 +/- 0.0808605 (43), Significance: 1.07351
\end{verbatim}
\end{frame}

\section{Conclusion}
\begin{frame}{Conclusion}
\begin{itemize}
\item Conclusion:
\begin{itemize}
\item We should see significant excess if the signal exist
\item We can exclude the signal if they are not exist
\end{itemize}
\item Few issues need to further study:
\begin{itemize}
\item The global maximum for significance is difficult to find.
\item The low statistics of background sample.
\item need to understand some of the optimization results.
\end{itemize}
\end{itemize}
\end{frame}

\section{Plan}
\begin{frame}{Plan}
\begin{itemize}
\item Dani and Samuel will agree on the SR selection (compare and agree on the final signal selection cuts).
\item Dongliang and Samuel will work on data/MC plots after pre-selection level (all lepton cuts, possibly also jets and b-jets cuts).
\item Plot the electron eta to make a decision about whether we use the electrons in the crack region.
\item Estimate charge flip BG and fake BG from data, both in SR and pre-selection level.
\end{itemize}
\end{frame}

\begin{frame}
\begin{center}
\huge
Backup
\end{center}
\end{frame}

\begin{frame}{Selection in run1 SR}
\includegraphics[width=\textwidth]{data/photo/SRcutrun1.png} \\
\url{https://arxiv.org/pdf/1501.07110.pdf}
\end{frame}

\begin{frame}
\frametitle{significance calculation}
\begin{itemize}
\item RooStats::NumberCountingUtils::BinomialExpZ(S,B,$\delta$B)
\item $\delta$B = 0.3
\end{itemize}
\end{frame}

\begin{frame}[fragile]{pre-selection}
\tiny
\begin{verbatim}
SR_SS_ee_1:
nCJet == 1
nBJet == 0
fabs(mll - 91.2) > 10

SR_SS_mumu_1:
nCJet == 1
nBJet == 0
fabs(eta1) < 2.4
fabs(eta2) < 2.4

SR_SS_emu_1:
nCJet == 1
nBJet == 0
fabs(eta1) < 2.5
fabs(eta2) < 2.5

SR_SS_ee_2:
nCJet == 2 || nCJet == 3
nBJet == 0
fabs(mll - 91.2) > 10

SR_SS_mumu_2:
nCJet == 2 || nCJet == 3
nBJet == 0
fabs(eta1) < 2.4
fabs(eta2) < 2.4

SR_SS_emu_2:
nCJet == 2 || nCJet == 3
nBJet == 0
fabs(eta1) < 2.5
fabs(eta2) < 2.5
\end{verbatim}
\end{frame}



\begin{frame}[fragile]{optimization}
\tiny
\begin{verbatim}
SR_SS_ee_1_opt: (175, 0):
30 <= pt1 < -1
20 <= pt2 < -1
50 <= ptll < 110
50 <= mTtwo < 100
0 <= fabs(eta1-eta2) < -1
55 <= METRel < -1
270 <= meff < 520
125 <= mtm < 140
0 <= mlj < 90

Zjets: 0 +/- 0 (0)
Wjets: 0 +/- 0 (0)
top: 0.00146966 +/- 0.00146966 (1)
VV: -0.0298588 +/- 0.0611412 (29)
Vgamma: 0 +/- 0 (0)
VVV: 0 +/- 0 (0)
Higgs: 0 +/- 0 (0)
Total BG: 0.00146966 +/- 0.00146966 (30)

Signal (175, 0): 1.54734 +/- 0.589634 (7), Significance: 3.97519
Signal (165, 35): 0.232796 +/- 0.232796 (1), Significance: 0.701642
Signal (400, 0): 0.0114094 +/- 0.0114094 (1), Significance: -1.45148
\end{verbatim}
\end{frame}

\begin{frame}[fragile]{optimization}
\tiny
\begin{verbatim}
SR_SS_mumu_1_opt: (175, 0):
30 <= pt1 < -1
20 <= pt2 < -1
0 <= ptll < -1
95 <= mTtwo < 110
0 <= fabs(eta1-eta2) < 1.5
85 <= METRel < 165
200 <= meff < -1
110 <= mtm < -1
0 <= mlj < 70

Zjets: 0 +/- 0 (0)
Wjets: 0 +/- 0 (0)
top: 0 +/- 0 (0)
VV: -0.0217223 +/- 0.118111 (16)
Vgamma: 0 +/- 0 (0)
VVV: 0.000545925 +/- 0.000545925 (1)
Higgs: 0 +/- 0 (0)
Total BG: 0.000545925 +/- 0.000545925 (17)

Signal (175, 0): 2.26167 +/- 0.771379 (9), Significance: 5.4749
Signal (165, 35): 0.635427 +/- 0.460394 (2), Significance: 2.33398
Signal (400, 0): 0.103048 +/- 0.032904 (10), Significance: 0.0355568
\end{verbatim}
\end{frame}

\begin{frame}[fragile]{optimization}
\tiny
\begin{verbatim}
SR_SS_emu_1_opt: (175, 0):
55 <= pt1 < 130
35 <= pt2 < 55
0 <= ptll < 100
0 <= mTtwo < -1
0 <= fabs(eta1-eta2) < 1.5
0 <= METRel < -1
230 <= meff < 270
115 <= mtm < 160
0 <= mlj < 60

Zjets: 0 +/- 0 (0)
Wjets: 0 +/- 0 (0)
top: 0.00396882 +/- 0.00396882 (1)
VV: -0.00383832 +/- 0.131183 (49)
Vgamma: 0 +/- 0 (0)
VVV: 0 +/- 0 (0)
Higgs: 0 +/- 0 (0)
Total BG: 0.00396882 +/- 0.00396882 (50)

Signal (175, 0): 1.64435 +/- 0.593237 (8), Significance: 3.74826
Signal (165, 35): 1.39318 +/- 0.630444 (5), Significance: 3.34622
Signal (400, 0): 0.0239312 +/- 0.0175233 (2), Significance: -1.13309
\end{verbatim}
\end{frame}

\begin{frame}[fragile]{optimization}
\tiny
\begin{verbatim}
SR_SS_ee_2_opt: (175, 0):
30 <= pt1 < -1
25 <= pt2 < 65
0 <= ptll < -1
85 <= mTtwo < 100
0 <= fabs(eta1-eta2) < 1
30 <= METRel < -1
240 <= meff < 370
110 <= mtm < -1
0 <= mlj < 120

Zjets: 0 +/- 0 (0)
Wjets: 0 +/- 0 (0)
top: -0.00256433 +/- 0.00256433 (1)
VV: 0.00600215 +/- 0.0922647 (10)
Vgamma: 0 +/- 0 (0)
VVV: 0 +/- 0 (0)
Higgs: 0.00259053 +/- 0.00185347 (2)
Total BG: 0.00859268 +/- 0.0922834 (13)

Signal (175, 0): 1.44604 +/- 0.762395 (4), Significance: 3.12548
Signal (165, 35): 0 +/- 0 (0), Significance: -1.81137
Signal (400, 0): 0.00872881 +/- 0.00872881 (1), Significance: -1.47961
\end{verbatim}
\end{frame}

\begin{frame}[fragile]{optimization}
\tiny
\begin{verbatim}
SR_SS_mumu_2_opt: (175, 0):
30 <= pt1 < -1
30 <= pt2 < -1
0 <= ptll < -1
70 <= mTtwo < 125
0 <= fabs(eta1-eta2) < 2
115 <= METRel < 140
200 <= meff < -1
0 <= mtm < -1
0 <= mlj < 105

Zjets: 0 +/- 0 (0)
Wjets: 0 +/- 0 (0)
top: 0.00193921 +/- 0.00391518 (2)
VV: -0.0174528 +/- 0.0415497 (9)
Vgamma: 0 +/- 0 (0)
VVV: 0 +/- 0 (0)
Higgs: 0 +/- 0 (0)
Total BG: 0.00193921 +/- 0.00391518 (11)

Signal (175, 0): 2.17909 +/- 0.782438 (8), Significance: 4.83057
Signal (165, 35): 0.772748 +/- 0.448994 (3), Significance: 2.36231
Signal (400, 0): 0.0460844 +/- 0.0240975 (4), Significance: -0.712443
\end{verbatim}
\end{frame}

\begin{frame}[fragile]{optimization}
\tiny
\begin{verbatim}
SR_SS_emu_2_opt: (175, 0):
50 <= pt1 < 195
30 <= pt2 < -1
0 <= ptll < -1
0 <= mTtwo < -1
0 <= fabs(eta1-eta2) < 1
145 <= METRel < -1
200 <= meff < -1
165 <= mtm < 270
0 <= mlj < 120

Zjets: 0 +/- 0 (0)
Wjets: -0.0546401 +/- 0.0546401 (1)
top: 0 +/- 0 (0)
VV: 0.0110805 +/- 0.0245428 (11)
Vgamma: 0 +/- 0 (0)
VVV: 0 +/- 0 (0)
Higgs: 0.00138988 +/- 0.00138988 (1)
Total BG: 0.0124704 +/- 0.0245822 (13)

Signal (175, 0): 2.12679 +/- 0.838426 (8), Significance: 3.89822
Signal (165, 35): 0.247514 +/- 0.247514 (1), Significance: 0.364401
Signal (400, 0): 0.271055 +/- 0.057169 (25), Significance: 0.454657
\end{verbatim}
\end{frame}

\begin{frame}[fragile]{optimization}
\tiny
\begin{verbatim}
SR_SS_ee_1_opt: (165, 35):
40 <= pt1 < 100
20 <= pt2 < -1
0 <= ptll < -1
30 <= mTtwo < 80
0 <= fabs(eta1-eta2) < -1
55 <= METRel < -1
200 <= meff < -1
125 <= mtm < 150
0 <= mlj < 80

Zjets: 0 +/- 0 (0)
Wjets: 0 +/- 0 (0)
top: 0.009579 +/- 0.00601021 (3)
VV: -0.768596 +/- 1.0458 (69)
Vgamma: 0 +/- 0 (0)
VVV: 0 +/- 0 (0)
Higgs: 0 +/- 0 (0)
Total BG: 0.009579 +/- 0.00601021 (72)

Signal (175, 0): 0.886845 +/- 0.542894 (3), Significance: 2.11352
Signal (165, 35): 2.18073 +/- 0.804241 (8), Significance: 4.09677
Signal (400, 0): 0 +/- 0 (0), Significance: -1.77438
\end{verbatim}
\end{frame}

\begin{frame}[fragile]{optimization}
\tiny
\begin{verbatim}
SR_SS_mumu_1_opt: (165, 35):
130 <= pt1 < 240
20 <= pt2 < -1
50 <= ptll < 210
0 <= mTtwo < -1
0 <= fabs(eta1-eta2) < 1.5
0 <= METRel < 90
200 <= meff < -1
100 <= mtm < 120
0 <= mlj < 90

Zjets: 0 +/- 0 (0)
Wjets: 0 +/- 0 (0)
top: -0.00318423 +/- 0.00318423 (1)
VV: -0.0176261 +/- 0.144285 (61)
Vgamma: 0 +/- 0 (0)
VVV: 0.000324421 +/- 0.000324421 (1)
Higgs: 0 +/- 0 (0)
Total BG: 0.000324421 +/- 0.000324421 (63)

Signal (175, 0): 0.413735 +/- 0.293086 (2), Significance: 1.73277
Signal (165, 35): 2.364 +/- 1.00339 (6), Significance: 5.83232
Signal (400, 0): 0.010161 +/- 0.010161 (1), Significance: -1.44055
\end{verbatim}
\end{frame}

\begin{frame}[fragile]{optimization}
\tiny
\begin{verbatim}
SR_SS_emu_1_opt: (165, 35):
25 <= pt1 < -1
155 <= pt2 < 195
0 <= ptll < -1
0 <= mTtwo < -1
0 <= fabs(eta1-eta2) < 1.5
0 <= METRel < -1
200 <= meff < -1
185 <= mtm < 215
0 <= mlj < 90

Zjets: 0 +/- 0 (0)
Wjets: 0 +/- 0 (0)
top: 0 +/- 0 (0)
VV: 0.000604354 +/- 0.034562 (3)
Vgamma: 0 +/- 0 (0)
VVV: 0 +/- 0 (0)
Higgs: 0 +/- 0 (0)
Total BG: 0.000604354 +/- 0.034562 (3)

Signal (175, 0): 0 +/- 0 (0), Significance: -2.64411
Signal (165, 35): 1.40795 +/- 1.40795 (1), Significance: 4.03712
Signal (400, 0): 0.0203205 +/- 0.0203205 (1), Significance: -1.11349
\end{verbatim}
\end{frame}

\begin{frame}[fragile]{optimization}
\tiny
\begin{verbatim}
SR_SS_ee_2_opt: (165, 35):
30 <= pt1 < -1
35 <= pt2 < 120
0 <= ptll < -1
0 <= mTtwo < -1
0 <= fabs(eta1-eta2) < -1
30 <= METRel < -1
100 <= meff < 480
130 <= mtm < 135
0 <= mlj < 120

Zjets: 0 +/- 0 (0)
Wjets: 0 +/- 0 (0)
top: 0 +/- 0 (0)
VV: -0.0156784 +/- 0.0770235 (19)
Vgamma: 0 +/- 0 (0)
VVV: 0.0109896 +/- 0.0109896 (1)
Higgs: 0 +/- 0 (0)
Total BG: 0.0109896 +/- 0.0109896 (20)

Signal (175, 0): 0.182675 +/- 0.182675 (1), Significance: 0.106581
Signal (165, 35): 1.17744 +/- 0.595486 (4), Significance: 2.59266
Signal (400, 0): 0.0118287 +/- 0.0118287 (1), Significance: -1.36878
\end{verbatim}
\end{frame}

\begin{frame}[fragile]{optimization}
\tiny
\begin{verbatim}
SR_SS_mumu_2_opt: (165, 35):
40 <= pt1 < 145
30 <= pt2 < 85
10 <= ptll < 140
25 <= mTtwo < 120
0 <= fabs(eta1-eta2) < 1.5
0 <= METRel < -1
250 <= meff < 540
60 <= mtm < 235
55 <= mlj < 120

Zjets: 0 +/- 0 (0)
Wjets: 0 +/- 0 (0)
top: 0.0855852 +/- 0.0212539 (41)
VV: 2.62522 +/- 0.234024 (644)
Vgamma: 0 +/- 0 (0)
VVV: 0.0252036 +/- 0.0147241 (3)
Higgs: 0.00578081 +/- 0.00495489 (8)
Total BG: 2.74179 +/- 0.2355 (696)

Signal (175, 0): 6.44166 +/- 1.33042 (25), Significance: 2.39701
Signal (165, 35): 9.13288 +/- 1.70313 (33), Significance: 3.21653
Signal (400, 0): 0.230802 +/- 0.0534235 (20), Significance: -0.140336
\end{verbatim}
\end{frame}

\begin{frame}[fragile]{optimization}
\tiny
\begin{verbatim}
SR_SS_emu_2_opt: (165, 35):
25 <= pt1 < 130
75 <= pt2 < 85
50 <= ptll < 100
0 <= mTtwo < -1
0 <= fabs(eta1-eta2) < 1.5
0 <= METRel < -1
200 <= meff < -1
135 <= mtm < 145
0 <= mlj < 120

Zjets: 0 +/- 0 (0)
Wjets: 0 +/- 0 (0)
top: 0 +/- 0 (0)
VV: 0 +/- 0 (0)
Vgamma: 0 +/- 0 (0)
VVV: 0.000283156 +/- 0.000283156 (1)
Higgs: 0 +/- 0 (0)
Total BG: 0.000283156 +/- 0.000283156 (1)

Signal (175, 0): 0 +/- 0 (0), Significance: -2.85797
Signal (165, 35): 1.25981 +/- 0.68261 (4), Significance: 3.98655
Signal (400, 0): 0 +/- 0 (0), Significance: -2.85797
\end{verbatim}
\end{frame}

\begin{frame}[fragile]{optimization}
\tiny
\begin{verbatim}
SR_SS_ee_1_opt: (400, 0):
30 <= pt1 < -1
25 <= pt2 < 115
70 <= ptll < 270
80 <= mTtwo < -1
0 <= fabs(eta1-eta2) < 3
55 <= METRel < -1
280 <= meff < -1
105 <= mtm < -1
0 <= mlj < 90

Zjets: 0 +/- 0 (0)
Wjets: 0 +/- 0 (0)
top: 0.0128592 +/- 0.00684218 (4)
VV: 0.0186209 +/- 0.277592 (48)
Vgamma: 0.0274795 +/- 0.0274795 (1)
VVV: 0.00286179 +/- 0.00249859 (3)
Higgs: 0 +/- 0 (0)
Total BG: 0.0618214 +/- 0.279044 (56)

Signal (175, 0): 1.60745 +/- 0.618266 (7), Significance: 2.44079
Signal (165, 35): 0 +/- 0 (0), Significance: -1.12164
Signal (400, 0): 1.00908 +/- 0.123451 (86), Significance: 1.62488
\end{verbatim}
\end{frame}

\begin{frame}[fragile]{optimization}
\tiny
\begin{verbatim}
SR_SS_mumu_1_opt: (400, 0):
30 <= pt1 < -1
25 <= pt2 < 130
0 <= ptll < -1
90 <= mTtwo < -1
0 <= fabs(eta1-eta2) < 3
100 <= METRel < -1
300 <= meff < -1
165 <= mtm < -1
25 <= mlj < 125

Zjets: 0 +/- 0 (0)
Wjets: 0 +/- 0 (0)
top: 0 +/- 0 (0)
VV: -0.00624522 +/- 0.128634 (25)
Vgamma: 0 +/- 0 (0)
VVV: 0.019289 +/- 0.0125299 (4)
Higgs: 0 +/- 0 (0)
Total BG: 0.019289 +/- 0.0125299 (29)

Signal (175, 0): 0.208462 +/- 0.208462 (1), Significance: 0.125016
Signal (165, 35): 0 +/- 0 (0), Significance: -1.53174
Signal (400, 0): 1.01548 +/- 0.135636 (76), Significance: 2.0986
\end{verbatim}
\end{frame}

\begin{frame}[fragile]{optimization}
\tiny
\begin{verbatim}
SR_SS_emu_1_opt: (400, 0):
30 <= pt1 < -1
30 <= pt2 < -1
0 <= ptll < -1
100 <= mTtwo < -1
0 <= fabs(eta1-eta2) < 3
95 <= METRel < -1
200 <= meff < -1
125 <= mtm < -1
20 <= mlj < 90

Zjets: 0.0150102 +/- 0.0150102 (1)
Wjets: 0 +/- 0 (0)
top: 0.01401 +/- 0.00888758 (5)
VV: -0.0134088 +/- 0.646095 (53)
Vgamma: 0.0513547 +/- 0.0324214 (3)
VVV: 0.0314345 +/- 0.0168197 (5)
Higgs: 0.00258108 +/- 0.00258108 (1)
Total BG: 0.11439 +/- 0.0405586 (68)

Signal (175, 0): 2.17543 +/- 0.664132 (11), Significance: 2.72168
Signal (165, 35): 0.933941 +/- 0.552928 (3), Significance: 1.26294
Signal (400, 0): 1.72387 +/- 0.161452 (145), Significance: 2.25105
\end{verbatim}
\end{frame}

\begin{frame}[fragile]{optimization}
\tiny
\begin{verbatim}
SR_SS_ee_2_opt: (400, 0):
30 <= pt1 < -1
25 <= pt2 < 185
30 <= ptll < -1
0 <= mTtwo < -1
0 <= fabs(eta1-eta2) < 3
40 <= METRel < -1
310 <= meff < -1
170 <= mtm < -1
55 <= mlj < 175

Zjets: 0.105361 +/- 0.105361 (1)
Wjets: 0.0417714 +/- 0.243132 (5)
top: 0.724428 +/- 0.384693 (31)
VV: 1.69476 +/- 0.685405 (263)
Vgamma: 0.111391 +/- 0.0563842 (4)
VVV: 0.0297791 +/- 0.0144241 (6)
Higgs: 0.0121984 +/- 0.00482783 (7)
Total BG: 2.71969 +/- 0.831501 (317)

Signal (175, 0): 0.914606 +/- 0.462678 (4), Significance: 0.219743
Signal (165, 35): 0.488084 +/- 0.345306 (2), Significance: -0.00078341
Signal (400, 0): 1.00798 +/- 0.130244 (82), Significance: 0.266422
\end{verbatim}
\end{frame}

\begin{frame}[fragile]{optimization}
\tiny
\begin{verbatim}
SR_SS_mumu_2_opt: (400, 0):
80 <= pt1 < 250
30 <= pt2 < -1
0 <= ptll < -1
85 <= mTtwo < -1
0 <= fabs(eta1-eta2) < 2
0 <= METRel < -1
100 <= meff < -1
160 <= mtm < -1
0 <= mlj < 185

Zjets: 0 +/- 0 (0)
Wjets: 0 +/- 0 (0)
top: 0.0126976 +/- 0.0135861 (8)
VV: -0.00805799 +/- 0.052725 (32)
Vgamma: 0 +/- 0 (0)
VVV: 0.0217483 +/- 0.0163369 (3)
Higgs: 0.00416985 +/- 0.00314653 (2)
Total BG: 0.0386157 +/- 0.0214797 (45)

Signal (175, 0): 0.876506 +/- 0.5176 (3), Significance: 1.59313
Signal (165, 35): 0.467114 +/- 0.330677 (2), Significance: 0.766513
Signal (400, 0): 1.02737 +/- 0.124713 (82), Significance: 1.84542
\end{verbatim}
\end{frame}

\begin{frame}[fragile]{optimization}
\tiny
\begin{verbatim}
SR_SS_emu_2_opt: (400, 0):
25 <= pt1 < 250
25 <= pt2 < 160
0 <= ptll < -1
65 <= mTtwo < -1
0 <= fabs(eta1-eta2) < 2
85 <= METRel < -1
310 <= meff < -1
170 <= mtm < -1
0 <= mlj < 115

Zjets: 0 +/- 0 (0)
Wjets: -0.0546401 +/- 0.0546401 (1)
top: 0.000356436 +/- 0.00452613 (2)
VV: -0.0364188 +/- 0.0817685 (25)
Vgamma: 0 +/- 0 (0)
VVV: 0 +/- 0 (0)
Higgs: 0.000146774 +/- 0.00186469 (2)
Total BG: 0.00050321 +/- 0.00489519 (30)

Signal (175, 0): 1.47596 +/- 0.564892 (7), Significance: 4.22299
Signal (165, 35): 1.68743 +/- 0.918036 (5), Significance: 4.59689
Signal (400, 0): 1.01567 +/- 0.115577 (91), Significance: 3.30047
\end{verbatim}
\end{frame}


\begin{frame}[fragile]
\frametitle{Signal sample}
\small
Sample Name(p2972 tag):
\tiny
\begin{verbatim}
mc15_13TeV.993820.MGPy8EG_A14N13LO_C1N2_Wh_2L_175_0.merge.DAOD_SUSY2.e5678_a766_a821_r7676_p2949_p2972
mc15_13TeV.993821.MGPy8EG_A14N13LO_C1N2_Wh_2L_165_35.merge.DAOD_SUSY2.e5678_a766_a821_r7676_p2949_p2972
mc15_13TeV.993822.MGPy8EG_A14N13LO_C1N2_Wh_2L_400_0.merge.DAOD_SUSY2.e5678_a766_a821_r7676_p2949_p2972
\end{verbatim}
\end{frame}

\begin{frame}[fragile]
\frametitle{Data}
\small
use both 2015 and 2016 data (3212.96 + 32861.6) /pb
\tiny
\begin{verbatim}
GRL:
GoodRunsLists/data16_13TeV/20161101/physics_25ns_20.7.xml
GoodRunsLists/data15_13TeV/20160720/physics_25ns_20.7.xml
\end{verbatim}
\end{frame}

\begin{frame}{MC BG}
p-tag: p2949
\end{frame}

\begin{frame}[fragile]
\small
Trigger list:\\
\scriptsize
\begin{verbatim}
2015
HLT_2e12_lhloose_L12EM10VH
HLT_e17_lhloose_mu14
HLT_mu18_mu8noL1

2016(A-D3)
HLT_2e17_lhvloose_nod0
HLT_e17_lhloose_nod0_mu14
HLT_mu20_mu8noL1

2016(D3-)
HLT_2e17_lhvloose_nod0
HLT_e17_lhloose_nod0_mu14
HLT_mu22_mu8noL1
\end{verbatim}
\end{frame}

\begin{frame}{Object Definitions}
\small
Tool: AnalysisBase 2.4.31, SUSYTools-00-08-60\\

\centering
\begin{table}
\small
\begin{tabularx}{\textwidth}{p{1.5cm} | p{3cm} | p{3cm} | p{3cm}}
& \textbf{Electron} & \textbf{Muon} & \textbf{Jet}\\
\hline
\textbf{Baseline}
& - $p_T>10$ GeV \newline - $|\eta^{cluster}| < 2.47$ \newline - LooseAndBLayerLLH
& - $p_T>10$ GeV \newline - $|\eta| < 2.7$ \newline - Medium
& - $p_T>20$ GeV \\
\hline
\textbf{Signal}
& - $p_T > 25$ GeV \newline - $|\eta^{cluster}| < 2.47$ \newline - TightLLH \newline - GradientLoose \newline - $|z_0 \sin \theta| < 0.5$mm \newline - $|d_0/\sigma_{d_0}| < 5$
& - $p_T > 25$ GeV \newline - $|\eta| < 2.7$ \newline - Medium \newline - GradientLoose \newline - $|z_0 \sin \theta| < 0.5$mm \newline - $|d_0/\sigma_{d_0}| < 3$
& - $p_T > 20$ GeV \newline - $|\eta|<2.8$ \newline \newline - $|JVT| > 0.59$ \newline if $p_T < 60$ GeV \newline and $|\eta| < 2.4$
\end{tabularx}
\end{table}

\raggedright
Selection:
\begin{itemize}
%\item Trigger selection
\item Exactly 2 baseline leptons and exactly 2 signal leptons
\end{itemize}

\tiny
Note: \\
Pileup reweighting is applied. \\
Scale factor for reconstruction, isolation, ID and trigger is applied.
\end{frame}

\begin{frame}
\frametitle{Definition of jets}
\normalsize
\begin{itemize}
\item Central jets: $\pt>20$ GeV, $|\eta|<2.4$, no b-tagged
\item B-jets: b-tagged
\end{itemize}
\end{frame}

\begin{frame}
\frametitle{definition of variables}
\normalsize
\begin{itemize}
\item HT: Sum of the $p_T$ of all signal jets and the two leptons.
\item R2 = MET / (MET + pt1 + pt2)
\item l12\_dPhi: difference in phi between the two leptons.
\item l12\_MET\_dPhi: difference in phi between MET and the sum of 4-momentum of the two leptons.
\end{itemize}
\end{frame}

%\begin{frame}{Expected number of events \\ For SR\_SS\_ee\_1}
\vspace{5mm}
\begin{tabular}{|c|c|c|}
\hline
& Number of events & Significance \\
\hline
Z+jets & $18.9\pm19.8$ & \\
\hline
W+jets & $3.3\pm2.1$ & \\
\hline
top & $69.1\pm5.8$ & \\
\hline
VV & $14.1\pm2.0$ & \\
\hline
V$+\gamma$ & $12.2\pm5.5$ & \\
\hline
VVV & $0.4\pm0.1$ & \\
\hline
Total BG & $117.9\pm21.6$ & \\
\hline
Signal (175, 0) & $7.1\pm1.2$ &$-0.009$\\
\hline
Signal (165, 35) & $2.4\pm0.4$ &$-0.134$\\
\hline
Signal (400, 0) & $9.8\pm0.8$ &$0.062$\\
\hline

\end{tabular}
\end{frame}

\begin{frame}{Expected number of events \\ For SR\_SS\_mumu\_1}
\vspace{5mm}
\begin{tabular}{|c|c|c|}
\hline
& Number of events & Significance \\
\hline
VV & $6.8\pm0.7$ & \\
\hline
V$+\gamma$ & $0.0\pm0.0$ & \\
\hline
Total BG & $6.8\pm0.7$ & \\
\hline
Signal (400, 380) & $0.1\pm0.0$ &$-0.201$\\
\hline
Signal (500, 450) & $0.4\pm0.0$ &$-0.106$\\
\hline
Signal (400, 300) & $2.6\pm0.2$ &$0.522$\\
\hline
Signal (400, 200) & $2.1\pm0.2$ &$0.380$\\
\hline
Signal (400, 100) & $1.7\pm0.3$ &$0.262$\\
\hline

\end{tabular}
\end{frame}

\begin{frame}{Expected number of events \\ For SR\_SS\_emu\_1}
\vspace{5mm}
\begin{tabular}{|c|c|c|}
\hline
& Number of events & Significance \\
\hline
Z+jets & $0.1\pm0.0$ & \\
\hline
W+jets & $5.0\pm2.4$ & \\
\hline
top & $39.9\pm3.6$ & \\
\hline
VV & $14.9\pm1.8$ & \\
\hline
V$+\gamma$ & $1.5\pm0.5$ & \\
\hline
VVV & $0.4\pm0.1$ & \\
\hline
Total BG & $61.9\pm4.7$ & \\
\hline
Signal (175, 0) & $8.1\pm1.2$ &$0.191$\\
\hline
Signal (165, 35) & $5.5\pm0.6$ &$0.068$\\
\hline
Signal (400, 0) & $15.0\pm1.1$ &$0.501$\\
\hline

\end{tabular}
\end{frame}

\begin{frame}{Expected number of events \\ For SR\_SS\_ee\_2}
\vspace{5mm}
\begin{tabular}{|c|c|c|}
\hline
& Number of events & Significance \\
\hline
Z+jets & $1.9\pm2.0$ & \\
\hline
W+jets & $0.5\pm0.5$ & \\
\hline
top & $12.4\pm1.9$ & \\
\hline
VV & $6.3\pm2.6$ & \\
\hline
V$+\gamma$ & $2.2\pm1.1$ & \\
\hline
VVV & $0.1\pm0.0$ & \\
\hline
Total BG & $23.5\pm3.9$ & \\
\hline
Signal (175, 0) & $1.9\pm0.4$ &$0.020$\\
\hline
Signal (165, 35) & $0.9\pm0.2$ &$-0.103$\\
\hline
Signal (400, 0) & $4.4\pm0.6$ &$0.294$\\
\hline

\end{tabular}
\end{frame}

\begin{frame}{Expected number of events \\ For SR\_SS\_mumu\_2}
\vspace{5mm}
\begin{tabular}{|c|c|c|}
\hline
& Number of events & Significance \\
\hline
Z+jets & $89.3\pm27.3$ & \\
\hline
W+jets & $1.1\pm0.6$ & \\
\hline
top & $10.7\pm1.5$ & \\
\hline
VV & $16.3\pm0.7$ & \\
\hline
V$+\gamma$ & $2.5\pm1.3$ & \\
\hline
VVV & $0.3\pm0.1$ & \\
\hline
Higgs & $6.5\pm3.7$ & \\
\hline
Total BG & $126.6\pm27.7$ & \\
\hline
Signal (175, 0) & $25.4\pm3.4$ &$0.413$\\
\hline
Signal (165, 35) & $21.3\pm2.6$ &$0.318$\\
\hline
Signal (400, 0) & $1.4\pm0.1$ &$-0.163$\\
\hline

\end{tabular}
\end{frame}

\begin{frame}{Expected number of events \\ For SR\_SS\_emu\_2}
\vspace{5mm}
\begin{tabular}{|c|c|c|}
\hline
& Number of events & Significance \\
\hline
Z+jets & $0.2\pm0.1$ & \\
\hline
W+jets & $0.8\pm0.5$ & \\
\hline
top & $14.6\pm2.3$ & \\
\hline
VV & $6.5\pm0.7$ & \\
\hline
V$+\gamma$ & $1.1\pm0.8$ & \\
\hline
VVV & $0.2\pm0.1$ & \\
\hline
Higgs & $1.3\pm0.6$ & \\
\hline
Total BG & $24.7\pm2.6$ & \\
\hline
Signal (175, 0) & $12.8\pm1.9$ &$1.078$\\
\hline
Signal (165, 35) & $12.6\pm2.0$ &$1.064$\\
\hline
Signal (400, 0) & $1.7\pm0.2$ &$-0.009$\\
\hline

\end{tabular}
\end{frame}


%\input{data/plot_SR_SS_run1.tex}

\end{document}
