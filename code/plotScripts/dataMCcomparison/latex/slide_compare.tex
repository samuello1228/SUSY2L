\documentclass[mathserif,serif]{beamer}
\usepackage{tabularx}
% \setbeamertemplate{footline}[frame number]
\useoutertheme{infolines}
\usepackage{slidesphysics}
\graphicspath{{../plot/}}

\title[SR optimization]{SS Analysis}
\author[]
{
Samuel Lo \inst{1}
\and
Yanjun Tu  \inst{1}
\and
Daniela Paredes \inst{1}
\and
Dongliang Zhang  \inst{2}
}
\institute[]
{
\inst{1}
The University of Hong Kong
\and
\inst{2}
University of Michigan
}
\date[]{\today}

\newcommand\Wider[2][2em]{%
\makebox[\linewidth][c]{%
\begin{minipage}{\dimexpr\textwidth+#1\relax}
\raggedright
\centering#2
\end{minipage}%
}%
}

\begin{document}
\frame{\titlepage}

\section{Introduction}
\begin{frame}{Update}
\begin{itemize}
\item use 13 signal points at the diagonal.
\item use MET.
\item Compare 2 SR optimizations:
\begin{itemize}
\item using MC to estimate charge flip BG. (listed in page 4)
\item using data-driven method to estimate charge flip BG. (using Gabriel's ntuple)
\end{itemize}
\end{itemize}
\end{frame}

\section{signal sample list}
\begin{frame}[fragile]{signal sample list}
13 signal points used at the diagonal

\tiny
\begin{verbatim}
(150,0)
(152.5,22.5)
(175,25)
(177.5,47.5)
(187.5,37.5)
(190,60)
(202.5,72.5)
(215,85)
(227.5,97.5)
(237.5,87.5)
(240,110)
(250,100)
(300,100)
\end{verbatim}
\end{frame}

\section{BG composition}
\begin{frame}[fragile]{BG composition}
\small
\begin{itemize}
\item ttV, multitop, VVV, Higgs
\item use the following MC as VV (excluding fake):

\tiny
\begin{verbatim}
361069.Sherpa_CT10_llvvjj_ss_EW4
361070.Sherpa_CT10_llvvjj_ss_EW6
361071.Sherpa_CT10_lllvjj_EW6
361072.Sherpa_CT10_lllljj_EW6
361073.Sherpa_CT10_ggllll
363490.Sherpa_221_NNPDF30NNLO_llll
363491.Sherpa_221_NNPDF30NNLO_lllv
\end{verbatim}

\small
\item fakes from Peter
\item use following MC as charge flip:
\begin{itemize}
\item Z+jets, Drell Yan
\item Zgamma (301535,301536,301899-301907)
\item $t\bar{t}$
\item Wt (410015,410016)
\item the following VV:

\tiny
\begin{verbatim}
361077.Sherpa_CT10_ggllvv
363356.Sherpa_221_NNPDF30NNLO_ZqqZll
363358.Sherpa_221_NNPDF30NNLO_WqqZll
363492.Sherpa_221_NNPDF30NNLO_llvv
\end{verbatim}

\end{itemize}
\end{itemize}
\end{frame}

\begin{frame}{optimization details}
\small
\begin{itemize}
\item The average significance over signal points used will be maximized in the optimization procedure.
\end{itemize}
The following constraints are applied to reduce the bias from low statistics:
\begin{itemize}
\small
\item The yields of groups of BG need to be positive, except for multi-top, V+gamma, Z+jets and W+jets.
\item At least 10 unweighted events in VV and ttV background.
\item If the yield for one group of BG is negative (only for multi-top, V+gamma, Z+jets and W+jets), 0 yield is used. But the statistical error still keep unchanged to calculate the statistical error of total BG.
\item nSig$>$1 and nSig/nSigError $>$ 2 for the first 10 signal mass points (393820 to 393829).
\item For $\pt^{l1}$, $\pt^{l1}$, $\pt^{ll}$, $m_{T2}$, $E_{\text{T}}^{\text{miss,rel}}$, $m_{\text{eff}}$ and $m_{\text{T}}^{\text{max}}$, only lower cut is applied.
\item For $|\Delta\eta_{ll}|$ and $m_{lj}$/$m_{ljj}$, only upper cut is applied.
\end{itemize}
\end{frame}

\section{Compare results}
\begin{frame}{Selection in SR}
\tiny
\begin{itemize}
\item $|\Delta\eta_{ll}|$ and $m_{\text{eff}}$ are optimized for new optimization.
\end{itemize}

\begin{table}[htbp]
\centering
\tiny

\begin{columns}

\begin{column}{0.5\textwidth}
\scalebox{0.8}{
\input{data/optimization/cut_table_SR_SS_opt_0_v5_R2_DD_MET.tex}
\end{tabular}
}
\caption{\tiny use MC as charge flip BG}
\end{column}

\begin{column}{0.5\textwidth}
\scalebox{0.8}{
\input{data/optimization/cut_table_SR_SS_opt_0_v6_s2_MET.tex}
\end{tabular}
}
\caption{\tiny use DD as charge flip BG}
\end{column}

\end{columns}
\end{table}
\end{frame}

\begin{frame}{Combined significance}
\tiny
\begin{itemize}
\item Combine the significance (only channels with $Z_n>0$ are combined)
\end{itemize}

\begin{columns}

\begin{column}{0.4\textwidth}
\begin{figure}
\includegraphics[width=0.7\textwidth]{../plot_v5_R2_DD_MET/combine_significance_0}
\caption{\tiny use MC as charge flip BG}
\end{figure}
\end{column}


\begin{column}{0.4\textwidth}
\begin{figure}
\includegraphics[width=0.7\textwidth]{../plot_v6_s2_MET/combine_significance_0}
\caption{\tiny use DD as charge flip BG}
\end{figure}
\end{column}

\end{columns}

\begin{itemize}
\item .
\end{itemize}
\end{frame}

\begin{frame}{Yields and significance in SR}

\begin{columns}

\begin{column}{0.4\textwidth}
\begin{figure}
\includegraphics[width=0.7\textwidth]{../plot_v5_R2_DD_MET/SR_SS_opt}
\caption{\tiny use MC as charge flip BG}
\end{figure}
\end{column}

\begin{column}{0.4\textwidth}
\begin{figure}
\includegraphics[width=0.7\textwidth]{../plot_v6_s2_MET/SR_SS_opt}
\caption{\tiny use DD as charge flip BG}
\end{figure}
\end{column}

\end{columns}

\end{frame}

%\section{Conclusion}
%\begin{frame}{Conclusion}
%\begin{itemize}
%\item .
%\end{itemize}
%\end{frame}

\section{SRjet1 and SRjet23}
\begin{frame}
\begin{center}
\huge
More plots for SR optimization using DD as charge flip BG
\end{center}
\end{frame}

\begin{frame}{signal yield plot}
\Wider[5em]{
\includegraphics[width=0.33\textwidth]{nSig_SR_SS_ee_1_opt_0}
\includegraphics[width=0.33\textwidth]{nSig_SR_SS_mumu_1_opt_0}
\includegraphics[width=0.33\textwidth]{nSig_SR_SS_emu_1_opt_0} \\
\includegraphics[width=0.33\textwidth]{nSig_SR_SS_ee_2_opt_0}
\includegraphics[width=0.33\textwidth]{nSig_SR_SS_mumu_2_opt_0}
\includegraphics[width=0.33\textwidth]{nSig_SR_SS_emu_2_opt_0}
}
\end{frame}

\begin{frame}{significance plot}
\Wider{
\includegraphics[width=0.33\textwidth]{significance_SR_SS_ee_1_opt_0}
\includegraphics[width=0.33\textwidth]{significance_SR_SS_mumu_1_opt_0}
\includegraphics[width=0.33\textwidth]{significance_SR_SS_emu_1_opt_0} \\
\includegraphics[width=0.33\textwidth]{significance_SR_SS_ee_2_opt_0}
\includegraphics[width=0.33\textwidth]{significance_SR_SS_mumu_2_opt_0}
\includegraphics[width=0.33\textwidth]{significance_SR_SS_emu_2_opt_0}
}
\end{frame}


\subsection{Yields table}
\begin{frame}{optimization}
\tiny
SR\_SS\_ee\_1\_opt: (175, 0): \\
$\pt^{l1}$ $\geq 35$ \\
$\pt^{l2}$ $\geq 30$ \\
$\pt^{ll}$ $\geq 50$ \\
$m_{T2}$ $\geq 65$ \\
$E_{\text{T}}^{\text{miss,rel}}$ $\geq 55$ \\
$m_{\text{eff}}$ $\geq 230$ \\
$m_{\text{T}}^{\text{max}}$ $\geq 75$ \\
$m_{lj}$/$m_{ljj}$ $<90$ \\

\begin{tabular}{|c|c|c|}
\hline
& Number of events & Significance \\
\hline
Z+jets & $0.000000\pm0.000000$ (0) & \\
\hline
W+jets & $0.008016\pm0.008016$ (1) & \\
\hline
$t\bar{t}$ & $0.220227\pm0.220227$ (1) & \\
\hline
single top & $0.000000\pm0.000000$ (0) & \\
\hline
$t\bar{t}+V$ & $0.030249\pm0.011923$ (11) & \\
\hline
multi top & $0.000000\pm0.000000$ (0) & \\
\hline
VV & $-0.414568\pm1.055392$ (125) & \\
\hline
V$+\gamma$ & $0.052589\pm0.037224$ (2) & \\
\hline
VVV & $0.045041\pm0.017974$ (10) & \\
\hline
Higgs & $0.001162\pm0.002303$ (3) & \\
\hline
Total BG & $0.357284\pm0.224545$ (153) & \\
\hline
Signal (175, 0) & $4.60\pm1.09$ (19) &$3.69$\\
\hline
Signal (165, 35) & $2.27\pm0.92$ (7) &$2.09$\\
\hline
Signal (400, 0) & $1.22\pm0.13$ (106) &$1.13$\\
\hline

\end{tabular}
\end{frame}

\begin{frame}{optimization}
\tiny
SR\_SS\_mumu\_1\_opt: (162.5,12.5): \\
$\pt^{l1}$ $\geq 50$ \\
$\pt^{l2}$ $\geq 20$ \\
$\pt^{ll}$ $\geq 20$ \\
$m_{T2}$ $\geq 15$ \\
$E_{\text{T}}^{\text{miss,rel}}$ $\geq 25$ \\
$m_{\text{eff}}$ $\geq 200$ \\
$m_{\text{T}}^{\text{max}}$ $\geq 105$ \\

\begin{tabular}{|c|c|c|}
\hline
& Number of events & Significance \\
\hline
VV & $6.900575\pm0.773596$ (1230) & \\
\hline
Higgs & $1.074043\pm0.619368$ (4) & \\
\hline
W+jets & $0.306715\pm0.306715$ (1) & \\
\hline
VVV & $0.302947\pm0.055987$ (45) & \\
\hline
$t\bar{t}$ & $0.268577\pm0.268577$ (1) & \\
\hline
single top & $0.235893\pm0.235893$ (1) & \\
\hline
$t\bar{t}+V$ & $0.091507\pm0.024986$ (26) & \\
\hline
V$+\gamma$ & $0.000000\pm0.000000$ (0) & \\
\hline
multi top & $0.000000\pm0.000000$ (0) & \\
\hline
Z+jets & $0.000000\pm0.000000$ (0) & \\
\hline
Total BG & $9.180256\pm1.098944$ (1308) & \\
\hline
(162.5,12.5) & $4.66\pm0.82$ (42) &$0.88$\\
\hline
(175.0,0.0) & $4.57\pm0.84$ (34) &$0.86$\\
\hline
(450.0,0.0) & $0.15\pm0.08$ (4) &$-0.16$\\
\hline

\end{tabular}
\end{frame}

\begin{frame}{optimization}
\tiny
SR\_SS\_emu\_1\_opt: (175, 0): \\
$\pt^{l1}$ $\geq 45$ \\
$\pt^{l2}$ $\geq 25$ \\
$\pt^{ll}$ $\geq 70$ \\
$m_{T2}$ $\geq 70$ \\
$|\Delta\eta_{ll}|$ $<1.5$ \\
$E_{\text{T}}^{\text{miss,rel}}$ $\geq 100$ \\
$m_{\text{eff}}$ $\geq 200$ \\
$m_{\text{T}}^{\text{max}}$ $\geq 70$ \\
$m_{lj}$/$m_{ljj}$ $<90$ \\

\begin{tabular}{|c|c|c|}
\hline
& Number of events & Significance \\
\hline
Z+jets & $0.015010\pm0.015010$ (1) & \\
\hline
W+jets & $0.007757\pm0.007757$ (1) & \\
\hline
$t\bar{t}$ & $0.273398\pm0.273398$ (1) & \\
\hline
single top & $0.000000\pm0.000000$ (0) & \\
\hline
$t\bar{t}+V$ & $0.005722\pm0.016653$ (11) & \\
\hline
multi top & $0.000000\pm0.000000$ (0) & \\
\hline
VV & $1.228071\pm0.320612$ (185) & \\
\hline
V$+\gamma$ & $0.425192\pm0.294809$ (3) & \\
\hline
VVV & $0.109876\pm0.030820$ (16) & \\
\hline
Higgs & $0.006179\pm0.003624$ (3) & \\
\hline
Total BG & $2.071205\pm0.515729$ (221) & \\
\hline
Signal (175, 0) & $7.59\pm2.34$ (25) &$2.95$\\
\hline
Signal (165, 35) & $6.36\pm1.37$ (23) &$2.54$\\
\hline
Signal (400, 0) & $1.75\pm0.16$ (145) &$0.68$\\
\hline

\end{tabular}
\end{frame}

\begin{frame}{optimization}
\tiny
SR\_SS\_ee\_2\_opt: (175, 0): \\
$\pt^{l1}$ $\geq 35$ \\
$\pt^{l2}$ $\geq 25$ \\
$\pt^{ll}$ $\geq 40$ \\
$m_{T2}$ $\geq 60$ \\
$E_{\text{T}}^{\text{miss,rel}}$ $\geq 20$ \\
$m_{\text{eff}}$ $\geq 100$ \\
$m_{\text{T}}^{\text{max}}$ $\geq 110$ \\
$m_{lj}$/$m_{ljj}$ $<120$ \\

\begin{tabular}{|c|c|c|}
\hline
& Number of events & Significance \\
\hline
Z+jets & $0.116844\pm0.116844$ (1) & \\
\hline
W+jets & $0.177670\pm0.245104$ (2) & \\
\hline
$t\bar{t}$ & $0.000000\pm0.000000$ (0) & \\
\hline
single top & $0.000000\pm0.000000$ (0) & \\
\hline
$t\bar{t}+V$ & $0.046698\pm0.016930$ (21) & \\
\hline
multi top & $0.000000\pm0.000000$ (0) & \\
\hline
VV & $1.136181\pm0.261530$ (320) & \\
\hline
V$+\gamma$ & $0.773702\pm0.458459$ (4) & \\
\hline
VVV & $0.055400\pm0.020261$ (10) & \\
\hline
Higgs & $0.527745\pm0.523065$ (7) & \\
\hline
Total BG & $2.834240\pm0.791584$ (365) & \\
\hline
Signal (175, 0) & $6.07\pm1.48$ (22) &$2.04$\\
\hline
Signal (165, 35) & $4.47\pm1.13$ (16) &$1.53$\\
\hline
Signal (400, 0) & $0.66\pm0.11$ (49) &$0.03$\\
\hline

\end{tabular}
\end{frame}

\begin{frame}{optimization}
\tiny
SR\_SS\_mumu\_2\_opt: (162.5,12.5): \\
$\pt^{l1}$ $\geq 25$ \\
$\pt^{l2}$ $\geq 30$ \\
$\pt^{ll}$ $\geq 30$ \\
$m_{T2}$ $\geq 20$ \\
$|\Delta\eta_{ll}|$ $<1.5$ \\
$E_{\text{T}}^{\text{miss,rel}}$ $\geq 35$ \\
$m_{\text{eff}}$ $\geq 220$ \\
$m_{\text{T}}^{\text{max}}$ $\geq 80$ \\
$m_{lj}$/$m_{ljj}$ $<120$ \\

\begin{tabular}{|c|c|c|}
\hline
& Number of events & Significance \\
\hline
Z+jets & $0.000000\pm0.000000$ (0) & \\
\hline
W+jets & $0.000000\pm0.000000$ (0) & \\
\hline
$t\bar{t}$ & $0.000000\pm0.000000$ (0) & \\
\hline
single top & $0.000000\pm0.000000$ (0) & \\
\hline
$t\bar{t}+V$ & $0.034675\pm0.018342$ (25) & \\
\hline
multi top & $0.000000\pm0.000000$ (0) & \\
\hline
VV & $2.623277\pm0.282610$ (600) & \\
\hline
V$+\gamma$ & $0.000000\pm0.000000$ (0) & \\
\hline
VVV & $0.092981\pm0.056192$ (5) & \\
\hline
Higgs & $0.392183\pm0.393018$ (5) & \\
\hline
Total BG & $3.143116\pm0.487674$ (635) & \\
\hline
Signal (175, 0) & $11.39\pm1.93$ (41) &$3.65$\\
\hline
Signal (165, 35) & $6.82\pm1.50$ (24) &$2.38$\\
\hline
Signal (400, 0) & $0.94\pm0.11$ (78) &$0.20$\\
\hline

\end{tabular}
\end{frame}

\begin{frame}{optimization}
\tiny
SR\_SS\_emu\_2\_opt: (162.5,12.5): \\
$\pt^{l1}$ $\geq 20$ \\
$\pt^{l2}$ $\geq 20$ \\
$m_{T2}$ $\geq 85$ \\
$|\Delta\eta_{ll}|$ $<2$ \\
$E_{\text{T}}^{\text{miss,rel}}$ $\geq 90$ \\
$m_{\text{T}}^{\text{max}}$ $\geq 130$ \\

\begin{tabular}{|c|c|c|}
\hline
& Number of events & Significance \\
\hline
VV & $2.656941\pm0.689645$ (860) & \\
\hline
single top & $0.416387\pm0.416387$ (1) & \\
\hline
$t\bar{t}+V$ & $0.271615\pm0.038882$ (108) & \\
\hline
$t\bar{t}$ & $0.264376\pm0.264376$ (1) & \\
\hline
VVV & $0.250505\pm0.051110$ (36) & \\
\hline
W+jets & $0.090438\pm0.079085$ (2) & \\
\hline
V$+\gamma$ & $0.041288\pm0.041288$ (1) & \\
\hline
Higgs & $0.012972\pm0.007562$ (18) & \\
\hline
Z+jets & $0.004904\pm0.004904$ (1) & \\
\hline
multi top & $0.000000\pm0.000000$ (0) & \\
\hline
Total BG & $4.009427\pm0.855013$ (1028) & \\
\hline
(162.5,12.5) & $4.05\pm0.70$ (38) &$1.21$\\
\hline
(175.0,0.0) & $4.22\pm0.78$ (32) &$1.26$\\
\hline
(450.0,0.0) & $0.55\pm0.14$ (17) &$-0.03$\\
\hline

\end{tabular}
\end{frame}


\subsection{``N-1'' plots}
\begin{frame}{$\pt^{l1}$}
\Wider{
\includegraphics[width=0.33\textwidth]{pt1_SR_SS_ee_1_opt_0}
\includegraphics[width=0.33\textwidth]{pt1_SR_SS_mumu_1_opt_0}
\includegraphics[width=0.33\textwidth]{pt1_SR_SS_emu_1_opt_0} \\
\includegraphics[width=0.33\textwidth]{pt1_SR_SS_ee_2_opt_0}
\includegraphics[width=0.33\textwidth]{pt1_SR_SS_mumu_2_opt_0}
\includegraphics[width=0.33\textwidth]{pt1_SR_SS_emu_2_opt_0}
}
\end{frame}

\begin{frame}{$\pt^{l2}$}
\Wider{
\includegraphics[width=0.33\textwidth]{pt2_SR_SS_ee_1_opt_0}
\includegraphics[width=0.33\textwidth]{pt2_SR_SS_mumu_1_opt_0}
\includegraphics[width=0.33\textwidth]{pt2_SR_SS_emu_1_opt_0} \\
\includegraphics[width=0.33\textwidth]{pt2_SR_SS_ee_2_opt_0}
\includegraphics[width=0.33\textwidth]{pt2_SR_SS_mumu_2_opt_0}
\includegraphics[width=0.33\textwidth]{pt2_SR_SS_emu_2_opt_0}
}
\end{frame}


\begin{frame}{$m_{ll}$}
\Wider{
\includegraphics[width=0.33\textwidth]{mll_SR_SS_ee_1_opt_0}
\includegraphics[width=0.33\textwidth]{mll_SR_SS_mumu_1_opt_0}
\includegraphics[width=0.33\textwidth]{mll_SR_SS_emu_1_opt_0} \\
\includegraphics[width=0.33\textwidth]{mll_SR_SS_ee_2_opt_0}
\includegraphics[width=0.33\textwidth]{mll_SR_SS_mumu_2_opt_0}
\includegraphics[width=0.33\textwidth]{mll_SR_SS_emu_2_opt_0}
}
\end{frame}

\begin{frame}{$|\Delta\eta_{ll}|$}
\Wider{
\includegraphics[width=0.33\textwidth]{dEta_SR_SS_ee_1_opt_0}
\includegraphics[width=0.33\textwidth]{dEta_SR_SS_mumu_1_opt_0}
\includegraphics[width=0.33\textwidth]{dEta_SR_SS_emu_1_opt_0} \\
\includegraphics[width=0.33\textwidth]{dEta_SR_SS_ee_2_opt_0}
\includegraphics[width=0.33\textwidth]{dEta_SR_SS_mumu_2_opt_0}
\includegraphics[width=0.33\textwidth]{dEta_SR_SS_emu_2_opt_0}
}
\end{frame}

\begin{frame}{$E_{\text{T}}^{\text{miss,rel}}$}
\Wider{
\includegraphics[width=0.33\textwidth]{METRel_SR_SS_ee_1_opt_0}
\includegraphics[width=0.33\textwidth]{METRel_SR_SS_mumu_1_opt_0}
\includegraphics[width=0.33\textwidth]{METRel_SR_SS_emu_1_opt_0} \\
\includegraphics[width=0.33\textwidth]{METRel_SR_SS_ee_2_opt_0}
\includegraphics[width=0.33\textwidth]{METRel_SR_SS_mumu_2_opt_0}
\includegraphics[width=0.33\textwidth]{METRel_SR_SS_emu_2_opt_0}
}
\end{frame}

\begin{frame}{$m_{\text{eff}}$}
\Wider{
\includegraphics[width=0.33\textwidth]{meff_SR_SS_ee_1_opt_0}
\includegraphics[width=0.33\textwidth]{meff_SR_SS_mumu_1_opt_0}
\includegraphics[width=0.33\textwidth]{meff_SR_SS_emu_1_opt_0} \\
\includegraphics[width=0.33\textwidth]{meff_SR_SS_ee_2_opt_0}
\includegraphics[width=0.33\textwidth]{meff_SR_SS_mumu_2_opt_0}
\includegraphics[width=0.33\textwidth]{meff_SR_SS_emu_2_opt_0}
}
\end{frame}

\begin{frame}{$m_{\text{T}}^{\text{max}}$}
\Wider{
\includegraphics[width=0.33\textwidth]{mtm_SR_SS_ee_1_opt_0}
\includegraphics[width=0.33\textwidth]{mtm_SR_SS_mumu_1_opt_0}
\includegraphics[width=0.33\textwidth]{mtm_SR_SS_emu_1_opt_0} \\
\includegraphics[width=0.33\textwidth]{mtm_SR_SS_ee_2_opt_0}
\includegraphics[width=0.33\textwidth]{mtm_SR_SS_mumu_2_opt_0}
\includegraphics[width=0.33\textwidth]{mtm_SR_SS_emu_2_opt_0}
}
\end{frame}

\begin{frame}{$m_{lj}$/$m_{ljj}$}
\Wider{
\includegraphics[width=0.33\textwidth]{mlj_SR_SS_ee_1_opt_0}
\includegraphics[width=0.33\textwidth]{mlj_SR_SS_mumu_1_opt_0}
\includegraphics[width=0.33\textwidth]{mlj_SR_SS_emu_1_opt_0} \\
\includegraphics[width=0.33\textwidth]{mlj_SR_SS_ee_2_opt_0}
\includegraphics[width=0.33\textwidth]{mlj_SR_SS_mumu_2_opt_0}
\includegraphics[width=0.33\textwidth]{mlj_SR_SS_emu_2_opt_0}
}
\end{frame}


\begin{frame}{$\pt^{ll}$}
\Wider{
\includegraphics[width=0.33\textwidth]{ptll_SR_SS_ee_1_opt_0}
\includegraphics[width=0.33\textwidth]{ptll_SR_SS_mumu_1_opt_0}
\includegraphics[width=0.33\textwidth]{ptll_SR_SS_emu_1_opt_0} \\
\includegraphics[width=0.33\textwidth]{ptll_SR_SS_ee_2_opt_0}
\includegraphics[width=0.33\textwidth]{ptll_SR_SS_mumu_2_opt_0}
\includegraphics[width=0.33\textwidth]{ptll_SR_SS_emu_2_opt_0}
}
\end{frame}

\begin{frame}{$m_{T2}$}
\Wider{
\includegraphics[width=0.33\textwidth]{mTtwo_SR_SS_ee_1_opt_0}
\includegraphics[width=0.33\textwidth]{mTtwo_SR_SS_mumu_1_opt_0}
\includegraphics[width=0.33\textwidth]{mTtwo_SR_SS_emu_1_opt_0} \\
\includegraphics[width=0.33\textwidth]{mTtwo_SR_SS_ee_2_opt_0}
\includegraphics[width=0.33\textwidth]{mTtwo_SR_SS_mumu_2_opt_0}
\includegraphics[width=0.33\textwidth]{mTtwo_SR_SS_emu_2_opt_0}
}
\end{frame}

\subsection{``N'' plots}
\begin{frame}{$E_{\text{T}}^{\text{miss}}$}
\Wider{
\includegraphics[width=0.33\textwidth]{MET_SR_SS_ee_1_opt_0}
\includegraphics[width=0.33\textwidth]{MET_SR_SS_mumu_1_opt_0}
\includegraphics[width=0.33\textwidth]{MET_SR_SS_emu_1_opt_0} \\
\includegraphics[width=0.33\textwidth]{MET_SR_SS_ee_2_opt_0}
\includegraphics[width=0.33\textwidth]{MET_SR_SS_mumu_2_opt_0}
\includegraphics[width=0.33\textwidth]{MET_SR_SS_emu_2_opt_0}
}
\end{frame}

\begin{frame}{$m_{\text{T}}^{l1}$}
\Wider{
\includegraphics[width=0.33\textwidth]{mt1_SR_SS_ee_1_opt_0}
\includegraphics[width=0.33\textwidth]{mt1_SR_SS_mumu_1_opt_0}
\includegraphics[width=0.33\textwidth]{mt1_SR_SS_emu_1_opt_0} \\
\includegraphics[width=0.33\textwidth]{mt1_SR_SS_ee_2_opt_0}
\includegraphics[width=0.33\textwidth]{mt1_SR_SS_mumu_2_opt_0}
\includegraphics[width=0.33\textwidth]{mt1_SR_SS_emu_2_opt_0}
}
\end{frame}

\begin{frame}{$m_{\text{T}}^{l2}$}
\Wider{
\includegraphics[width=0.33\textwidth]{mt2_SR_SS_ee_1_opt_0}
\includegraphics[width=0.33\textwidth]{mt2_SR_SS_mumu_1_opt_0}
\includegraphics[width=0.33\textwidth]{mt2_SR_SS_emu_1_opt_0} \\
\includegraphics[width=0.33\textwidth]{mt2_SR_SS_ee_2_opt_0}
\includegraphics[width=0.33\textwidth]{mt2_SR_SS_mumu_2_opt_0}
\includegraphics[width=0.33\textwidth]{mt2_SR_SS_emu_2_opt_0}
}
\end{frame}

\begin{frame}{$\pt$ of the leading jet}
\Wider{
\includegraphics[width=0.33\textwidth]{jetpt_SR_SS_ee_1_opt_0}
\includegraphics[width=0.33\textwidth]{jetpt_SR_SS_mumu_1_opt_0}
\includegraphics[width=0.33\textwidth]{jetpt_SR_SS_emu_1_opt_0} \\
\includegraphics[width=0.33\textwidth]{jetpt_SR_SS_ee_2_opt_0}
\includegraphics[width=0.33\textwidth]{jetpt_SR_SS_mumu_2_opt_0}
\includegraphics[width=0.33\textwidth]{jetpt_SR_SS_emu_2_opt_0}
}
\end{frame}

\begin{frame}{$\eta$ of the leading jet}
\Wider{
\includegraphics[width=0.33\textwidth]{jeteta_SR_SS_ee_1_opt_0}
\includegraphics[width=0.33\textwidth]{jeteta_SR_SS_mumu_1_opt_0}
\includegraphics[width=0.33\textwidth]{jeteta_SR_SS_emu_1_opt_0} \\
\includegraphics[width=0.33\textwidth]{jeteta_SR_SS_ee_2_opt_0}
\includegraphics[width=0.33\textwidth]{jeteta_SR_SS_mumu_2_opt_0}
\includegraphics[width=0.33\textwidth]{jeteta_SR_SS_emu_2_opt_0}
}
\end{frame}

\begin{frame}{Number of jets}
\Wider{
\includegraphics[width=0.33\textwidth]{nJet_SR_SS_ee_1_opt_0}
\includegraphics[width=0.33\textwidth]{nJet_SR_SS_mumu_1_opt_0}
\includegraphics[width=0.33\textwidth]{nJet_SR_SS_emu_1_opt_0} \\
\includegraphics[width=0.33\textwidth]{nJet_SR_SS_ee_2_opt_0}
\includegraphics[width=0.33\textwidth]{nJet_SR_SS_mumu_2_opt_0}
\includegraphics[width=0.33\textwidth]{nJet_SR_SS_emu_2_opt_0}
}
\end{frame}

\begin{frame}{Number of b-jets}
\Wider{
\includegraphics[width=0.33\textwidth]{nBJet_SR_SS_ee_1_opt_0}
\includegraphics[width=0.33\textwidth]{nBJet_SR_SS_mumu_1_opt_0}
\includegraphics[width=0.33\textwidth]{nBJet_SR_SS_emu_1_opt_0} \\
\includegraphics[width=0.33\textwidth]{nBJet_SR_SS_ee_2_opt_0}
\includegraphics[width=0.33\textwidth]{nBJet_SR_SS_mumu_2_opt_0}
\includegraphics[width=0.33\textwidth]{nBJet_SR_SS_emu_2_opt_0}
}
\end{frame}

\begin{frame}{Number of central jets}
\Wider{
\includegraphics[width=0.33\textwidth]{nCJet_SR_SS_ee_1_opt_0}
\includegraphics[width=0.33\textwidth]{nCJet_SR_SS_mumu_1_opt_0}
\includegraphics[width=0.33\textwidth]{nCJet_SR_SS_emu_1_opt_0} \\
\includegraphics[width=0.33\textwidth]{nCJet_SR_SS_ee_2_opt_0}
\includegraphics[width=0.33\textwidth]{nCJet_SR_SS_mumu_2_opt_0}
\includegraphics[width=0.33\textwidth]{nCJet_SR_SS_emu_2_opt_0}
}
\end{frame}

\begin{frame}{$|\Delta\phi_{ll}|$}
\Wider{
\includegraphics[width=0.33\textwidth]{l12_dPhi_SR_SS_ee_1_opt_0}
\includegraphics[width=0.33\textwidth]{l12_dPhi_SR_SS_mumu_1_opt_0}
\includegraphics[width=0.33\textwidth]{l12_dPhi_SR_SS_emu_1_opt_0} \\
\includegraphics[width=0.33\textwidth]{l12_dPhi_SR_SS_ee_2_opt_0}
\includegraphics[width=0.33\textwidth]{l12_dPhi_SR_SS_mumu_2_opt_0}
\includegraphics[width=0.33\textwidth]{l12_dPhi_SR_SS_emu_2_opt_0}
}
\end{frame}

\begin{frame}{$|\Delta\phi_{ll,\text{MET}}|$}
\Wider{
\includegraphics[width=0.33\textwidth]{l12_MET_dPhi_SR_SS_ee_1_opt_0}
\includegraphics[width=0.33\textwidth]{l12_MET_dPhi_SR_SS_mumu_1_opt_0}
\includegraphics[width=0.33\textwidth]{l12_MET_dPhi_SR_SS_emu_1_opt_0} \\
\includegraphics[width=0.33\textwidth]{l12_MET_dPhi_SR_SS_ee_2_opt_0}
\includegraphics[width=0.33\textwidth]{l12_MET_dPhi_SR_SS_mumu_2_opt_0}
\includegraphics[width=0.33\textwidth]{l12_MET_dPhi_SR_SS_emu_2_opt_0}
}
\end{frame}

\begin{frame}{$|\Delta\phi_{\text{jet0,MET}}|$}
\Wider{
\includegraphics[width=0.33\textwidth]{jet0_MET_dPhi_SR_SS_ee_1_opt_0}
\includegraphics[width=0.33\textwidth]{jet0_MET_dPhi_SR_SS_mumu_1_opt_0}
\includegraphics[width=0.33\textwidth]{jet0_MET_dPhi_SR_SS_emu_1_opt_0} \\
\includegraphics[width=0.33\textwidth]{jet0_MET_dPhi_SR_SS_ee_2_opt_0}
\includegraphics[width=0.33\textwidth]{jet0_MET_dPhi_SR_SS_mumu_2_opt_0}
\includegraphics[width=0.33\textwidth]{jet0_MET_dPhi_SR_SS_emu_2_opt_0}
}
\end{frame}

\begin{frame}{$m_{jj}$}
\Wider{
\includegraphics[width=0.33\textwidth]{mjj_SR_SS_ee_1_opt_0}
\includegraphics[width=0.33\textwidth]{mjj_SR_SS_mumu_1_opt_0}
\includegraphics[width=0.33\textwidth]{mjj_SR_SS_emu_1_opt_0} \\
\includegraphics[width=0.33\textwidth]{mjj_SR_SS_ee_2_opt_0}
\includegraphics[width=0.33\textwidth]{mjj_SR_SS_mumu_2_opt_0}
\includegraphics[width=0.33\textwidth]{mjj_SR_SS_emu_2_opt_0}
}
\end{frame}

\begin{frame}{$\eta^{l1}$}
\Wider{
\includegraphics[width=0.33\textwidth]{eta1_SR_SS_ee_1_opt_0}
\includegraphics[width=0.33\textwidth]{eta1_SR_SS_mumu_1_opt_0}
\includegraphics[width=0.33\textwidth]{eta1_SR_SS_emu_1_opt_0} \\
\includegraphics[width=0.33\textwidth]{eta1_SR_SS_ee_2_opt_0}
\includegraphics[width=0.33\textwidth]{eta1_SR_SS_mumu_2_opt_0}
\includegraphics[width=0.33\textwidth]{eta1_SR_SS_emu_2_opt_0}
}
\end{frame}

\begin{frame}{$\eta^{l2}$}
\Wider{
\includegraphics[width=0.33\textwidth]{eta2_SR_SS_ee_1_opt_0}
\includegraphics[width=0.33\textwidth]{eta2_SR_SS_mumu_1_opt_0}
\includegraphics[width=0.33\textwidth]{eta2_SR_SS_emu_1_opt_0} \\
\includegraphics[width=0.33\textwidth]{eta2_SR_SS_ee_2_opt_0}
\includegraphics[width=0.33\textwidth]{eta2_SR_SS_mumu_2_opt_0}
\includegraphics[width=0.33\textwidth]{eta2_SR_SS_emu_2_opt_0}
}
\end{frame}

\begin{frame}{$\phi^{l1}$}
\Wider{
\includegraphics[width=0.33\textwidth]{phi1_SR_SS_ee_1_opt_0}
\includegraphics[width=0.33\textwidth]{phi1_SR_SS_mumu_1_opt_0}
\includegraphics[width=0.33\textwidth]{phi1_SR_SS_emu_1_opt_0} \\
\includegraphics[width=0.33\textwidth]{phi1_SR_SS_ee_2_opt_0}
\includegraphics[width=0.33\textwidth]{phi1_SR_SS_mumu_2_opt_0}
\includegraphics[width=0.33\textwidth]{phi1_SR_SS_emu_2_opt_0}
}
\end{frame}




\section{Pre-selection plots}
\subsection{Yields}
\begin{frame}{For SRee1}
\vspace{5mm}
\begin{tabular}{|c|c|c|}
\hline
& Number of events & Significance \\
\hline
Z+jets & $1041.127228\pm153.507203$ (1003) & \\
\hline
W+jets & $255.071259\pm85.080870$ (223) & \\
\hline
V$+\gamma$ & $242.385849\pm13.376917$ (768) & \\
\hline
VV & $217.914223\pm5.009122$ (28876) & \\
\hline
$t\bar{t}$ & $11.460234\pm3.301990$ (25) & \\
\hline
Higgs & $9.826902\pm2.298310$ (105) & \\
\hline
single top & $5.452925\pm1.307231$ (29) & \\
\hline
VVV & $2.095896\pm0.165902$ (325) & \\
\hline
$t\bar{t}+V$ & $0.391007\pm0.058742$ (137) & \\
\hline
multi top & $0.000000\pm0.000000$ (0) & \\
\hline
Total BG & $1785.725523\pm176.139644$ (31491) & \\
\hline
(162.5,12.5) & $23.53\pm1.71$ (224) &$-0.13$\\
\hline
(175.0,0.0) & $15.57\pm1.48$ (122) &$-0.15$\\
\hline
(450.0,0.0) & $0.29\pm0.11$ (8) &$-0.18$\\
\hline

\end{tabular}
\end{frame}

\begin{frame}{For SRmumu1}
\vspace{5mm}
\begin{tabular}{|c|c|c|}
\hline
& Number of events & Significance \\
\hline
Z+jets & $0.662430\pm4.520373$ (37) & \\
\hline
W+jets & $25.506329\pm12.605377$ (74) & \\
\hline
$t\bar{t}$ & $4.310779\pm1.135067$ (15) & \\
\hline
single top & $2.688027\pm0.824791$ (14) & \\
\hline
$t\bar{t}+V$ & $1.022556\pm0.089410$ (339) & \\
\hline
multi top & $0.000000\pm0.000000$ (0) & \\
\hline
VV & $187.623191\pm4.043243$ (30036) & \\
\hline
V$+\gamma$ & $0.000000\pm0.000000$ (0) & \\
\hline
VVV & $2.572765\pm0.185894$ (412) & \\
\hline
Higgs & $13.885622\pm2.604483$ (53) & \\
\hline
Total BG & $238.271700\pm14.299359$ (30980) & \\
\hline
Signal (175, 0) & $66.38\pm5.01$ (245) &$0.81$\\
\hline
Signal (165, 35) & $76.39\pm5.40$ (238) &$0.94$\\
\hline
Signal (400, 0) & $3.10\pm0.22$ (248) &$-0.12$\\
\hline

\end{tabular}
\end{frame}

\begin{frame}{For SRemu1}
\vspace{5mm}
\begin{tabular}{|c|c|c|}
\hline
& Number of events & Significance \\
\hline
Z+jets & $93.750413\pm40.567822$ (255) & \\
\hline
W+jets & $324.962656\pm118.077523$ (461) & \\
\hline
$t\bar{t}$ & $20.441992\pm4.040177$ (50) & \\
\hline
single top & $7.393506\pm1.488672$ (39) & \\
\hline
$t\bar{t}+V$ & $0.864343\pm0.107271$ (318) & \\
\hline
multi top & $0.000000\pm0.000000$ (0) & \\
\hline
VV & $447.990089\pm5.885438$ (64261) & \\
\hline
V$+\gamma$ & $209.621932\pm12.089898$ (654) & \\
\hline
VVV & $5.195837\pm0.279168$ (832) & \\
\hline
Higgs & $14.832354\pm2.588111$ (111) & \\
\hline
Total BG & $1125.053123\pm125.674900$ (66981) & \\
\hline
(162.5,12.5) & $54.88\pm2.67$ (523) &$-0.01$\\
\hline
(202.5,72.5) & $24.65\pm1.88$ (212) &$-0.10$\\
\hline
(450.0,0.0) & $0.93\pm0.20$ (25) &$-0.18$\\
\hline

\end{tabular}
\end{frame}

\begin{frame}{For SRee2}
\vspace{5mm}
\begin{tabular}{|c|c|c|}
\hline
& Number of events & Significance \\
\hline
Z+jets & $2970.265358\pm207.526984$ (14201) & \\
\hline
W+jets & $124.760580\pm38.882738$ (692) & \\
\hline
$t\bar{t}$ & $121.614945\pm6.835866$ (435) & \\
\hline
single top & $13.131676\pm1.357751$ (110) & \\
\hline
$t\bar{t}+V$ & $3.904214\pm0.169408$ (1716) & \\
\hline
multi top & $0.002473\pm0.001833$ (4) & \\
\hline
VV & $297.522864\pm8.633744$ (62258) & \\
\hline
V$+\gamma$ & $312.731511\pm13.196942$ (1722) & \\
\hline
VVV & $1.957936\pm0.165767$ (408) & \\
\hline
Higgs & $14.638054\pm2.855301$ (561) & \\
\hline
Total BG & $3860.529612\pm211.860328$ (82107) & \\
\hline
Signal (175, 0) & $36.05\pm3.65$ (140) &$-0.13$\\
\hline
Signal (165, 35) & $51.30\pm5.61$ (151) &$-0.12$\\
\hline
Signal (400, 0) & $3.25\pm0.22$ (273) &$-0.17$\\
\hline

\end{tabular}
\end{frame}

\begin{frame}{For SRmumu2}
\vspace{5mm}
\begin{tabular}{|c|c|c|}
\hline
& Number of events & Significance \\
\hline
Z+jets & $30.911743\pm16.623014$ (118) & \\
\hline
W+jets & $75.577828\pm16.297410$ (307) & \\
\hline
$t\bar{t}$ & $52.409081\pm4.246995$ (182) & \\
\hline
single top & $6.337801\pm1.795583$ (32) & \\
\hline
$t\bar{t}+V$ & $4.906009\pm0.187888$ (1921) & \\
\hline
multi top & $0.000238\pm0.000238$ (1) & \\
\hline
VV & $326.965631\pm4.030877$ (85579) & \\
\hline
V$+\gamma$ & $0.015268\pm0.010936$ (2) & \\
\hline
VVV & $3.488119\pm0.228909$ (632) & \\
\hline
Higgs & $20.014418\pm4.481653$ (196) & \\
\hline
Total BG & $520.626134\pm24.486982$ (88970) & \\
\hline
Signal (175, 0) & $103.17\pm6.73$ (353) &$0.56$\\
\hline
Signal (165, 35) & $112.31\pm6.52$ (364) &$0.62$\\
\hline
Signal (400, 0) & $4.87\pm0.26$ (407) &$-0.13$\\
\hline

\end{tabular}
\end{frame}

\begin{frame}{For SRemu2}
\vspace{5mm}
\begin{tabular}{|c|c|c|}
\hline
& Number of events & Significance \\
\hline
Z+jets & $116.969752\pm23.521205$ (489) & \\
\hline
W+jets & $192.186476\pm76.464783$ (696) & \\
\hline
$t\bar{t}$ & $75.588083\pm5.272688$ (256) & \\
\hline
single top & $8.870611\pm1.460928$ (61) & \\
\hline
$t\bar{t}+V$ & $7.747451\pm0.250025$ (3158) & \\
\hline
multi top & $0.000873\pm0.000438$ (4) & \\
\hline
VV & $549.830395\pm4.903205$ (142285) & \\
\hline
V$+\gamma$ & $177.947444\pm13.158731$ (690) & \\
\hline
VVV & $5.524547\pm0.279397$ (1080) & \\
\hline
Higgs & $27.387550\pm3.643723$ (397) & \\
\hline
Total BG & $1162.053181\pm81.490224$ (149116) & \\
\hline
(162.5,12.5) & $68.66\pm3.10$ (646) &$0.05$\\
\hline
(202.5,72.5) & $36.08\pm2.61$ (281) &$-0.05$\\
\hline
(450.0,0.0) & $1.47\pm0.22$ (46) &$-0.17$\\
\hline

\end{tabular}
\end{frame}


\subsection{Plots}
\begin{frame}{For SR\_SS\_pre \\ $\pt^{l1}$}
\Wider[5em]{
\includegraphics[width=0.33\textwidth]{pt1_SR_SS_ee_1_pre}
\includegraphics[width=0.33\textwidth]{pt1_SR_SS_mumu_1_pre}
\includegraphics[width=0.33\textwidth]{pt1_SR_SS_emu_1_pre} \\
\includegraphics[width=0.33\textwidth]{pt1_SR_SS_ee_2_pre}
\includegraphics[width=0.33\textwidth]{pt1_SR_SS_mumu_2_pre}
\includegraphics[width=0.33\textwidth]{pt1_SR_SS_emu_2_pre}
}
\end{frame}

\begin{frame}{For SR\_SS\_pre \\ $\pt^{l2}$}
\Wider[5em]{
\includegraphics[width=0.33\textwidth]{pt2_SR_SS_ee_1_pre}
\includegraphics[width=0.33\textwidth]{pt2_SR_SS_mumu_1_pre}
\includegraphics[width=0.33\textwidth]{pt2_SR_SS_emu_1_pre} \\
\includegraphics[width=0.33\textwidth]{pt2_SR_SS_ee_2_pre}
\includegraphics[width=0.33\textwidth]{pt2_SR_SS_mumu_2_pre}
\includegraphics[width=0.33\textwidth]{pt2_SR_SS_emu_2_pre}
}
\end{frame}

\begin{frame}{For SR\_SS\_pre \\ $\eta^{l1}$}
\Wider[5em]{
\includegraphics[width=0.33\textwidth]{eta1_SR_SS_ee_1_pre}
\includegraphics[width=0.33\textwidth]{eta1_SR_SS_mumu_1_pre}
\includegraphics[width=0.33\textwidth]{eta1_SR_SS_emu_1_pre} \\
\includegraphics[width=0.33\textwidth]{eta1_SR_SS_ee_2_pre}
\includegraphics[width=0.33\textwidth]{eta1_SR_SS_mumu_2_pre}
\includegraphics[width=0.33\textwidth]{eta1_SR_SS_emu_2_pre}
}
\end{frame}

\begin{frame}{For SR\_SS\_pre \\ $\eta^{l2}$}
\Wider[5em]{
\includegraphics[width=0.33\textwidth]{eta2_SR_SS_ee_1_pre}
\includegraphics[width=0.33\textwidth]{eta2_SR_SS_mumu_1_pre}
\includegraphics[width=0.33\textwidth]{eta2_SR_SS_emu_1_pre} \\
\includegraphics[width=0.33\textwidth]{eta2_SR_SS_ee_2_pre}
\includegraphics[width=0.33\textwidth]{eta2_SR_SS_mumu_2_pre}
\includegraphics[width=0.33\textwidth]{eta2_SR_SS_emu_2_pre}
}
\end{frame}

\begin{frame}{For SR\_SS\_pre \\ $\phi^{l1}$}
\Wider[5em]{
\includegraphics[width=0.33\textwidth]{phi1_SR_SS_ee_1_pre}
\includegraphics[width=0.33\textwidth]{phi1_SR_SS_mumu_1_pre}
\includegraphics[width=0.33\textwidth]{phi1_SR_SS_emu_1_pre} \\
\includegraphics[width=0.33\textwidth]{phi1_SR_SS_ee_2_pre}
\includegraphics[width=0.33\textwidth]{phi1_SR_SS_mumu_2_pre}
\includegraphics[width=0.33\textwidth]{phi1_SR_SS_emu_2_pre}
}
\end{frame}

\begin{frame}{For SR\_SS\_pre \\ $m_{ll}$}
\Wider[5em]{
\includegraphics[width=0.33\textwidth]{mll_SR_SS_ee_1_pre}
\includegraphics[width=0.33\textwidth]{mll_SR_SS_mumu_1_pre}
\includegraphics[width=0.33\textwidth]{mll_SR_SS_emu_1_pre} \\
\includegraphics[width=0.33\textwidth]{mll_SR_SS_ee_2_pre}
\includegraphics[width=0.33\textwidth]{mll_SR_SS_mumu_2_pre}
\includegraphics[width=0.33\textwidth]{mll_SR_SS_emu_2_pre}
}
\end{frame}

\begin{frame}{For SR\_SS\_pre \\ $\pt^{ll}$}
\Wider[5em]{
\includegraphics[width=0.33\textwidth]{ptll_SR_SS_ee_1_pre}
\includegraphics[width=0.33\textwidth]{ptll_SR_SS_mumu_1_pre}
\includegraphics[width=0.33\textwidth]{ptll_SR_SS_emu_1_pre} \\
\includegraphics[width=0.33\textwidth]{ptll_SR_SS_ee_2_pre}
\includegraphics[width=0.33\textwidth]{ptll_SR_SS_mumu_2_pre}
\includegraphics[width=0.33\textwidth]{ptll_SR_SS_emu_2_pre}
}
\end{frame}

\begin{frame}{For SR\_SS\_pre \\ $E_{\text{T}}^{\text{miss}}$}
\Wider[5em]{
\includegraphics[width=0.33\textwidth]{MET_SR_SS_ee_1_pre}
\includegraphics[width=0.33\textwidth]{MET_SR_SS_mumu_1_pre}
\includegraphics[width=0.33\textwidth]{MET_SR_SS_emu_1_pre} \\
\includegraphics[width=0.33\textwidth]{MET_SR_SS_ee_2_pre}
\includegraphics[width=0.33\textwidth]{MET_SR_SS_mumu_2_pre}
\includegraphics[width=0.33\textwidth]{MET_SR_SS_emu_2_pre}
}
\end{frame}

\begin{frame}{For SR\_SS\_pre \\ $m_{T2}$}
\Wider[5em]{
\includegraphics[width=0.33\textwidth]{mTtwo_SR_SS_ee_1_pre}
\includegraphics[width=0.33\textwidth]{mTtwo_SR_SS_mumu_1_pre}
\includegraphics[width=0.33\textwidth]{mTtwo_SR_SS_emu_1_pre} \\
\includegraphics[width=0.33\textwidth]{mTtwo_SR_SS_ee_2_pre}
\includegraphics[width=0.33\textwidth]{mTtwo_SR_SS_mumu_2_pre}
\includegraphics[width=0.33\textwidth]{mTtwo_SR_SS_emu_2_pre}
}
\end{frame}

\begin{frame}{For SR\_SS\_pre \\ $m_{\text{T}}^{l1}$}
\Wider[5em]{
\includegraphics[width=0.33\textwidth]{mt1_SR_SS_ee_1_pre}
\includegraphics[width=0.33\textwidth]{mt1_SR_SS_mumu_1_pre}
\includegraphics[width=0.33\textwidth]{mt1_SR_SS_emu_1_pre} \\
\includegraphics[width=0.33\textwidth]{mt1_SR_SS_ee_2_pre}
\includegraphics[width=0.33\textwidth]{mt1_SR_SS_mumu_2_pre}
\includegraphics[width=0.33\textwidth]{mt1_SR_SS_emu_2_pre}
}
\end{frame}

\begin{frame}{For SR\_SS\_pre \\ $m_{\text{T}}^{l2}$}
\Wider[5em]{
\includegraphics[width=0.33\textwidth]{mt2_SR_SS_ee_1_pre}
\includegraphics[width=0.33\textwidth]{mt2_SR_SS_mumu_1_pre}
\includegraphics[width=0.33\textwidth]{mt2_SR_SS_emu_1_pre} \\
\includegraphics[width=0.33\textwidth]{mt2_SR_SS_ee_2_pre}
\includegraphics[width=0.33\textwidth]{mt2_SR_SS_mumu_2_pre}
\includegraphics[width=0.33\textwidth]{mt2_SR_SS_emu_2_pre}
}
\end{frame}

\begin{frame}{For SR\_SS\_pre \\ $\pt$ of the leading jet}
\Wider[5em]{
\includegraphics[width=0.33\textwidth]{jetpt_SR_SS_ee_1_pre}
\includegraphics[width=0.33\textwidth]{jetpt_SR_SS_mumu_1_pre}
\includegraphics[width=0.33\textwidth]{jetpt_SR_SS_emu_1_pre} \\
\includegraphics[width=0.33\textwidth]{jetpt_SR_SS_ee_2_pre}
\includegraphics[width=0.33\textwidth]{jetpt_SR_SS_mumu_2_pre}
\includegraphics[width=0.33\textwidth]{jetpt_SR_SS_emu_2_pre}
}
\end{frame}

\begin{frame}{For SR\_SS\_pre \\ $\eta$ of the leading jet}
\Wider[5em]{
\includegraphics[width=0.33\textwidth]{jeteta_SR_SS_ee_1_pre}
\includegraphics[width=0.33\textwidth]{jeteta_SR_SS_mumu_1_pre}
\includegraphics[width=0.33\textwidth]{jeteta_SR_SS_emu_1_pre} \\
\includegraphics[width=0.33\textwidth]{jeteta_SR_SS_ee_2_pre}
\includegraphics[width=0.33\textwidth]{jeteta_SR_SS_mumu_2_pre}
\includegraphics[width=0.33\textwidth]{jeteta_SR_SS_emu_2_pre}
}
\end{frame}

\begin{frame}{For SR\_SS\_pre \\ Number of jets}
\Wider[5em]{
\includegraphics[width=0.33\textwidth]{nJet_SR_SS_ee_1_pre}
\includegraphics[width=0.33\textwidth]{nJet_SR_SS_mumu_1_pre}
\includegraphics[width=0.33\textwidth]{nJet_SR_SS_emu_1_pre} \\
\includegraphics[width=0.33\textwidth]{nJet_SR_SS_ee_2_pre}
\includegraphics[width=0.33\textwidth]{nJet_SR_SS_mumu_2_pre}
\includegraphics[width=0.33\textwidth]{nJet_SR_SS_emu_2_pre}
}
\end{frame}

\begin{frame}{For SR\_SS\_pre \\ Number of b-jets}
\Wider[5em]{
\includegraphics[width=0.33\textwidth]{nBJet_SR_SS_ee_1_pre}
\includegraphics[width=0.33\textwidth]{nBJet_SR_SS_mumu_1_pre}
\includegraphics[width=0.33\textwidth]{nBJet_SR_SS_emu_1_pre} \\
\includegraphics[width=0.33\textwidth]{nBJet_SR_SS_ee_2_pre}
\includegraphics[width=0.33\textwidth]{nBJet_SR_SS_mumu_2_pre}
\includegraphics[width=0.33\textwidth]{nBJet_SR_SS_emu_2_pre}
}
\end{frame}

\begin{frame}{For SR\_SS\_pre \\ Number of central jets}
\Wider[5em]{
\includegraphics[width=0.33\textwidth]{nCJet_SR_SS_ee_1_pre}
\includegraphics[width=0.33\textwidth]{nCJet_SR_SS_mumu_1_pre}
\includegraphics[width=0.33\textwidth]{nCJet_SR_SS_emu_1_pre} \\
\includegraphics[width=0.33\textwidth]{nCJet_SR_SS_ee_2_pre}
\includegraphics[width=0.33\textwidth]{nCJet_SR_SS_mumu_2_pre}
\includegraphics[width=0.33\textwidth]{nCJet_SR_SS_emu_2_pre}
}
\end{frame}

\begin{frame}{For SR\_SS\_pre \\ $|\Delta\phi_{ll}|$}
\Wider[5em]{
\includegraphics[width=0.33\textwidth]{l12_dPhi_SR_SS_ee_1_pre}
\includegraphics[width=0.33\textwidth]{l12_dPhi_SR_SS_mumu_1_pre}
\includegraphics[width=0.33\textwidth]{l12_dPhi_SR_SS_emu_1_pre} \\
\includegraphics[width=0.33\textwidth]{l12_dPhi_SR_SS_ee_2_pre}
\includegraphics[width=0.33\textwidth]{l12_dPhi_SR_SS_mumu_2_pre}
\includegraphics[width=0.33\textwidth]{l12_dPhi_SR_SS_emu_2_pre}
}
\end{frame}

\begin{frame}{For SR\_SS\_pre \\ $|\Delta\phi_{ll,\text{MET}}|$}
\Wider[5em]{
\includegraphics[width=0.33\textwidth]{l12_MET_dPhi_SR_SS_ee_1_pre}
\includegraphics[width=0.33\textwidth]{l12_MET_dPhi_SR_SS_mumu_1_pre}
\includegraphics[width=0.33\textwidth]{l12_MET_dPhi_SR_SS_emu_1_pre} \\
\includegraphics[width=0.33\textwidth]{l12_MET_dPhi_SR_SS_ee_2_pre}
\includegraphics[width=0.33\textwidth]{l12_MET_dPhi_SR_SS_mumu_2_pre}
\includegraphics[width=0.33\textwidth]{l12_MET_dPhi_SR_SS_emu_2_pre}
}
\end{frame}

\begin{frame}{For SR\_SS\_pre \\ $|\Delta\phi_{\text{jet0,MET}}|$}
\Wider[5em]{
\includegraphics[width=0.33\textwidth]{jet0_MET_dPhi_SR_SS_ee_1_pre}
\includegraphics[width=0.33\textwidth]{jet0_MET_dPhi_SR_SS_mumu_1_pre}
\includegraphics[width=0.33\textwidth]{jet0_MET_dPhi_SR_SS_emu_1_pre} \\
\includegraphics[width=0.33\textwidth]{jet0_MET_dPhi_SR_SS_ee_2_pre}
\includegraphics[width=0.33\textwidth]{jet0_MET_dPhi_SR_SS_mumu_2_pre}
\includegraphics[width=0.33\textwidth]{jet0_MET_dPhi_SR_SS_emu_2_pre}
}
\end{frame}

\begin{frame}{For SR\_SS\_pre \\ $|\Delta\eta_{ll}|$}
\Wider[5em]{
\includegraphics[width=0.33\textwidth]{dEta_SR_SS_ee_1_pre}
\includegraphics[width=0.33\textwidth]{dEta_SR_SS_mumu_1_pre}
\includegraphics[width=0.33\textwidth]{dEta_SR_SS_emu_1_pre} \\
\includegraphics[width=0.33\textwidth]{dEta_SR_SS_ee_2_pre}
\includegraphics[width=0.33\textwidth]{dEta_SR_SS_mumu_2_pre}
\includegraphics[width=0.33\textwidth]{dEta_SR_SS_emu_2_pre}
}
\end{frame}

\begin{frame}{For SR\_SS\_pre \\ $E_{\text{T}}^{\text{miss,rel}}$}
\Wider[5em]{
\includegraphics[width=0.33\textwidth]{METRel_SR_SS_ee_1_pre}
\includegraphics[width=0.33\textwidth]{METRel_SR_SS_mumu_1_pre}
\includegraphics[width=0.33\textwidth]{METRel_SR_SS_emu_1_pre} \\
\includegraphics[width=0.33\textwidth]{METRel_SR_SS_ee_2_pre}
\includegraphics[width=0.33\textwidth]{METRel_SR_SS_mumu_2_pre}
\includegraphics[width=0.33\textwidth]{METRel_SR_SS_emu_2_pre}
}
\end{frame}

\begin{frame}{For SR\_SS\_pre \\ $m_{\text{eff}}$}
\Wider[5em]{
\includegraphics[width=0.33\textwidth]{meff_SR_SS_ee_1_pre}
\includegraphics[width=0.33\textwidth]{meff_SR_SS_mumu_1_pre}
\includegraphics[width=0.33\textwidth]{meff_SR_SS_emu_1_pre} \\
\includegraphics[width=0.33\textwidth]{meff_SR_SS_ee_2_pre}
\includegraphics[width=0.33\textwidth]{meff_SR_SS_mumu_2_pre}
\includegraphics[width=0.33\textwidth]{meff_SR_SS_emu_2_pre}
}
\end{frame}

\begin{frame}{For SR\_SS\_pre \\ $m_{\text{T}}^{\text{max}}$}
\Wider[5em]{
\includegraphics[width=0.33\textwidth]{mtm_SR_SS_ee_1_pre}
\includegraphics[width=0.33\textwidth]{mtm_SR_SS_mumu_1_pre}
\includegraphics[width=0.33\textwidth]{mtm_SR_SS_emu_1_pre} \\
\includegraphics[width=0.33\textwidth]{mtm_SR_SS_ee_2_pre}
\includegraphics[width=0.33\textwidth]{mtm_SR_SS_mumu_2_pre}
\includegraphics[width=0.33\textwidth]{mtm_SR_SS_emu_2_pre}
}
\end{frame}

\begin{frame}{For SR\_SS\_pre \\ $m_{lj}$/$m_{ljj}$}
\Wider[5em]{
\includegraphics[width=0.33\textwidth]{mlj_SR_SS_ee_1_pre}
\includegraphics[width=0.33\textwidth]{mlj_SR_SS_mumu_1_pre}
\includegraphics[width=0.33\textwidth]{mlj_SR_SS_emu_1_pre} \\
\includegraphics[width=0.33\textwidth]{mlj_SR_SS_ee_2_pre}
\includegraphics[width=0.33\textwidth]{mlj_SR_SS_mumu_2_pre}
\includegraphics[width=0.33\textwidth]{mlj_SR_SS_emu_2_pre}
}
\end{frame}

\begin{frame}{For SR\_SS\_pre \\ $m_{jj}$}
\Wider[5em]{
\includegraphics[width=0.33\textwidth]{mjj_SR_SS_ee_1_pre}
\includegraphics[width=0.33\textwidth]{mjj_SR_SS_mumu_1_pre}
\includegraphics[width=0.33\textwidth]{mjj_SR_SS_emu_1_pre} \\
\includegraphics[width=0.33\textwidth]{mjj_SR_SS_ee_2_pre}
\includegraphics[width=0.33\textwidth]{mjj_SR_SS_mumu_2_pre}
\includegraphics[width=0.33\textwidth]{mjj_SR_SS_emu_2_pre}
}
\end{frame}



\section*{Backup}
\begin{frame}
\begin{center}
\huge
Backup
\end{center}
\end{frame}

\begin{frame}
\small
SUSY scenario:\\
\begin{figure}
\includegraphics[width=0.6\textwidth]{data/photo/Wh.png}
\end{figure}
\end{frame}

\begin{frame}[fragile]
\frametitle{Signal sample}
\small
Sample Name(p2949 tag):
\tiny
\begin{verbatim}
mc15_13TeV.
393822.MGPy8EG_A14N23LO_C1N2_Wh_hall_162p5_12p5_2L7.merge.DAOD_SUSY2.e6153_a766_a821_r7676_p2949

mc15_13TeV.
393823.MGPy8EG_A14N23LO_C1N2_Wh_hall_175p0_0p0_2L7.merge.DAOD_SUSY2.e6153_a766_a821_r7676_p2949

mc15_13TeV.
393895.MGPy8EG_A14N23LO_C1N2_Wh_hall_450p0_0p0_2L7.merge.DAOD_SUSY2.e6153_a766_a821_r7676_p2949
\end{verbatim}
\end{frame}

\begin{frame}[fragile]
\frametitle{Data}
\small
use both 2015 and 2016 data (3212.96 + 32861.6) /pb
\tiny
\begin{verbatim}
GRL:
GoodRunsLists/data16_13TeV/20161101/physics_25ns_20.7.xml
GoodRunsLists/data15_13TeV/20160720/physics_25ns_20.7.xml
\end{verbatim}
\end{frame}

\begin{frame}{MC BG}
p-tag: p2949
\end{frame}

\begin{frame}[fragile]
\small
Trigger list:\\
\scriptsize
\begin{verbatim}
2015
HLT_2e12_lhloose_L12EM10VH
HLT_e17_lhloose_mu14
HLT_mu18_mu8noL1

2016(A-D3)
HLT_2e17_lhvloose_nod0
HLT_e17_lhloose_nod0_mu14
HLT_mu20_mu8noL1

2016(D3-)
HLT_2e17_lhvloose_nod0
HLT_e17_lhloose_nod0_mu14
HLT_mu22_mu8noL1
\end{verbatim}
\end{frame}

\begin{frame}{Object Definitions}
\small
Tool: AnalysisBase 2.4.31, SUSYTools-00-08-60\\

\centering
\begin{table}
\small
\begin{tabularx}{\textwidth}{p{1.5cm} | p{3cm} | p{3cm} | p{3cm}}
& \textbf{Electron} & \textbf{Muon} & \textbf{Jet}\\
\hline
\textbf{Baseline}
& - $p_T>10$ GeV \newline - $|\eta^{cluster}| < 2.47$ \newline - LooseAndBLayerLLH
& - $p_T>10$ GeV \newline - $|\eta| < 2.4$ \newline - Medium
& - $p_T>20$ GeV \\
\hline
\textbf{Signal}
& - $p_T > 20$ GeV \newline - $|\eta^{cluster}| < 2.47$ \newline - MediumLLH \newline - FixedCutTight \newline - $|z_0 \sin \theta| < 0.5$mm \newline - $|d_0/\sigma_{d_0}| < 5$
& - $p_T > 20$ GeV \newline - $|\eta| < 2.4$ \newline - Medium \newline - GradientLoose \newline - $|z_0 \sin \theta| < 0.5$mm \newline - $|d_0/\sigma_{d_0}| < 3$
& - $p_T > 20$ GeV \newline - $|\eta|<2.8$ \newline \newline - $|JVT| > 0.59$ \newline if $p_T < 60$ GeV \newline and $|\eta| < 2.4$
\end{tabularx}
\end{table}

\raggedright
Selection:
\begin{itemize}
\item Trigger requirement
\item Exactly 2 baseline leptons and exactly 2 signal leptons
\end{itemize}

\tiny
Note: \\
Pileup reweighting is applied. \\
Scale factor for reconstruction, isolation, ID and trigger is applied.
\end{frame}

\begin{frame}
\frametitle{Definition of jets}
\normalsize
\begin{itemize}
\item Central jets: $\pt>20$ GeV, $|\eta|<2.8$, no b-tagged
\begin{itemize}
\item Use new eta cut.
\end{itemize}
\item B-jets: b-tagged
\end{itemize}
\end{frame}

\begin{frame}
\frametitle{significance calculation}
\begin{itemize}
\item RooStats::NumberCountingUtils::BinomialExpZ(S,B,$\delta$B)
\item $\delta$B = sqrt((0.25)\^{}2 + (nBGError/nBG)\^{}2)
\begin{itemize}
\item use the same definition from Dani.
\end{itemize}
\end{itemize}
\end{frame}

\begin{frame}
\frametitle{definition of variables}
\normalsize
\begin{itemize}
\item HT: Sum of the $p_T$ of all signal jets and the two leptons.
\item R2 = MET / (MET + pt1 + pt2)
\item l12\_dPhi: difference in phi between the two leptons.
\item l12\_MET\_dPhi: difference in phi between MET and the sum of 4-momentum of the two leptons.
\end{itemize}
\end{frame}

\end{document}
