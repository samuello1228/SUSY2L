\documentclass[mathserif,serif]{beamer}
\usepackage{tabularx}
\setbeamertemplate{footline}[frame number]
% \useoutertheme{infolines}
\usepackage{slidesphysics}
\graphicspath{{../plot/}}

\title[]{Data/MC comparison}
\author[]
{
Samuel Lo \inst{1}
\and
Yanjun Tu  \inst{1}
\and
Dongliang Zhang  \inst{2}
}
\institute[]
{
\inst{1}
The University of Hong Kong
\and
\inst{2}
University of Michigan
}
\date[]{\today}

\newcommand\Wider[2][3em]{%
\makebox[\linewidth][c]{%
\begin{minipage}{\dimexpr\textwidth+#1\relax}
\raggedright
\centering#2
\end{minipage}%
}%
}

\begin{document}
\frame{\titlepage}

\begin{frame}{Outline}
\tableofcontents
\end{frame}

\section{Update}

\begin{frame}{Introduction}
\begin{itemize}
\item Cut flow comparison
\begin{itemize}
\item The weighted number get close (within 2\%) between Daniela and me for Zmumu background. But some small difference still need to be resolved.
\url{https://docs.google.com/spreadsheets/d/12-hpz\_X154YYrYVbQPT6TL5JLWWPZa40NhN7AhhBpdc/edit\#gid=1424837111}
\end{itemize}
\item SR Optimization
\begin{itemize}
\item Comparison for with/without charge flip tagger. The Zee BG is reduced by 6 times, while the signal has small change.
\url{https://its.cern.ch/jira/browse/SUSYEWKWH-14}
\item Comparison of signal and BG in the SR of run 1
\url{https://its.cern.ch/jira/projects/SUSYEWKWH/issues/SUSYEWKWH-9?filter=allopenissues}
\end{itemize}
\end{itemize}
\end{frame}

\section{CR\_SS\_run1}
\begin{frame}
\sectionpage
\end{frame}

\begin{frame}{Selection in run1 SR}
\includegraphics[width=\textwidth]{data/photo/SRcutrun1.png} \\
\url{https://arxiv.org/pdf/1501.07110.pdf}
\end{frame}

\begin{frame}{Expected number of events \\ For SR\_SS\_ee\_1}
\vspace{5mm}
\begin{tabular}{|c|c|c|}
\hline
& Number of events & Significance \\
\hline
Z+jets & $18.9\pm19.8$ & \\
\hline
W+jets & $3.3\pm2.1$ & \\
\hline
top & $69.1\pm5.8$ & \\
\hline
VV & $14.1\pm2.0$ & \\
\hline
V$+\gamma$ & $12.2\pm5.5$ & \\
\hline
VVV & $0.4\pm0.1$ & \\
\hline
Total BG & $117.9\pm21.6$ & \\
\hline
Signal (175, 0) & $7.1\pm1.2$ &$-0.009$\\
\hline
Signal (165, 35) & $2.4\pm0.4$ &$-0.134$\\
\hline
Signal (400, 0) & $9.8\pm0.8$ &$0.062$\\
\hline

\end{tabular}
\end{frame}

\begin{frame}{Expected number of events \\ For SR\_SS\_mumu\_1}
\vspace{5mm}
\begin{tabular}{|c|c|c|}
\hline
& Number of events & Significance \\
\hline
VV & $6.8\pm0.7$ & \\
\hline
V$+\gamma$ & $0.0\pm0.0$ & \\
\hline
Total BG & $6.8\pm0.7$ & \\
\hline
Signal (400, 380) & $0.1\pm0.0$ &$-0.201$\\
\hline
Signal (500, 450) & $0.4\pm0.0$ &$-0.106$\\
\hline
Signal (400, 300) & $2.6\pm0.2$ &$0.522$\\
\hline
Signal (400, 200) & $2.1\pm0.2$ &$0.380$\\
\hline
Signal (400, 100) & $1.7\pm0.3$ &$0.262$\\
\hline

\end{tabular}
\end{frame}

\begin{frame}{Expected number of events \\ For SR\_SS\_emu\_1}
\vspace{5mm}
\begin{tabular}{|c|c|c|}
\hline
& Number of events & Significance \\
\hline
Z+jets & $0.1\pm0.0$ & \\
\hline
W+jets & $5.0\pm2.4$ & \\
\hline
top & $39.9\pm3.6$ & \\
\hline
VV & $14.9\pm1.8$ & \\
\hline
V$+\gamma$ & $1.5\pm0.5$ & \\
\hline
VVV & $0.4\pm0.1$ & \\
\hline
Total BG & $61.9\pm4.7$ & \\
\hline
Signal (175, 0) & $8.1\pm1.2$ &$0.191$\\
\hline
Signal (165, 35) & $5.5\pm0.6$ &$0.068$\\
\hline
Signal (400, 0) & $15.0\pm1.1$ &$0.501$\\
\hline

\end{tabular}
\end{frame}

\begin{frame}{Expected number of events \\ For SR\_SS\_ee\_2}
\vspace{5mm}
\begin{tabular}{|c|c|c|}
\hline
& Number of events & Significance \\
\hline
Z+jets & $1.9\pm2.0$ & \\
\hline
W+jets & $0.5\pm0.5$ & \\
\hline
top & $12.4\pm1.9$ & \\
\hline
VV & $6.3\pm2.6$ & \\
\hline
V$+\gamma$ & $2.2\pm1.1$ & \\
\hline
VVV & $0.1\pm0.0$ & \\
\hline
Total BG & $23.5\pm3.9$ & \\
\hline
Signal (175, 0) & $1.9\pm0.4$ &$0.020$\\
\hline
Signal (165, 35) & $0.9\pm0.2$ &$-0.103$\\
\hline
Signal (400, 0) & $4.4\pm0.6$ &$0.294$\\
\hline

\end{tabular}
\end{frame}

\begin{frame}{Expected number of events \\ For SR\_SS\_mumu\_2}
\vspace{5mm}
\begin{tabular}{|c|c|c|}
\hline
& Number of events & Significance \\
\hline
Z+jets & $89.3\pm27.3$ & \\
\hline
W+jets & $1.1\pm0.6$ & \\
\hline
top & $10.7\pm1.5$ & \\
\hline
VV & $16.3\pm0.7$ & \\
\hline
V$+\gamma$ & $2.5\pm1.3$ & \\
\hline
VVV & $0.3\pm0.1$ & \\
\hline
Higgs & $6.5\pm3.7$ & \\
\hline
Total BG & $126.6\pm27.7$ & \\
\hline
Signal (175, 0) & $25.4\pm3.4$ &$0.413$\\
\hline
Signal (165, 35) & $21.3\pm2.6$ &$0.318$\\
\hline
Signal (400, 0) & $1.4\pm0.1$ &$-0.163$\\
\hline

\end{tabular}
\end{frame}

\begin{frame}{Expected number of events \\ For SR\_SS\_emu\_2}
\vspace{5mm}
\begin{tabular}{|c|c|c|}
\hline
& Number of events & Significance \\
\hline
Z+jets & $0.2\pm0.1$ & \\
\hline
W+jets & $0.8\pm0.5$ & \\
\hline
top & $14.6\pm2.3$ & \\
\hline
VV & $6.5\pm0.7$ & \\
\hline
V$+\gamma$ & $1.1\pm0.8$ & \\
\hline
VVV & $0.2\pm0.1$ & \\
\hline
Higgs & $1.3\pm0.6$ & \\
\hline
Total BG & $24.7\pm2.6$ & \\
\hline
Signal (175, 0) & $12.8\pm1.9$ &$1.078$\\
\hline
Signal (165, 35) & $12.6\pm2.0$ &$1.064$\\
\hline
Signal (400, 0) & $1.7\pm0.2$ &$-0.009$\\
\hline

\end{tabular}
\end{frame}


\input{data/plot_SR_SS_run1.tex}

\section{Conclusion}
\begin{frame}{Conclusion}
\begin{itemize}
\item ?
\end{itemize}
\end{frame}

\section{Plan}
\begin{frame}{Plan}
\begin{itemize}
\item BG composition need to be understood
\begin{itemize}
\item large contribution of Zmumu in mumu channel
\item large contribution of top in ee and emu channel
\end{itemize}
\item 2D optimization
\item Check with data in CR
\item Estimate charge flip BG from data
\item Estimate fake BG from data
\end{itemize}
\end{frame}

\begin{frame}
\begin{center}
\huge
Backup
\end{center}
\end{frame}

\begin{frame}[fragile]
\frametitle{Signal sample}
\small
Sample Name(p2972 tag):
\tiny
\begin{verbatim}
mc15_13TeV.993820.MGPy8EG_A14N13LO_C1N2_Wh_2L_175_0.merge.DAOD_SUSY2.e5678_a766_a821_r7676_p2949_p2972
mc15_13TeV.993821.MGPy8EG_A14N13LO_C1N2_Wh_2L_165_35.merge.DAOD_SUSY2.e5678_a766_a821_r7676_p2949_p2972
mc15_13TeV.993822.MGPy8EG_A14N13LO_C1N2_Wh_2L_400_0.merge.DAOD_SUSY2.e5678_a766_a821_r7676_p2949_p2972
\end{verbatim}
\end{frame}

\begin{frame}[fragile]
\frametitle{Data}
\small
use both 2015 and 2016 data (3212.96 + 32861.6) /pb
\tiny
\begin{verbatim}
GRL:
GoodRunsLists/data16_13TeV/20161101/physics_25ns_20.7.xml
GoodRunsLists/data15_13TeV/20160720/physics_25ns_20.7.xml
\end{verbatim}
\end{frame}

\begin{frame}{MC BG}
p-tag: p2949
\end{frame}

\begin{frame}[fragile]
\small
Trigger list:\\
\scriptsize
\begin{verbatim}
2015
HLT_2e12_lhloose_L12EM10VH
HLT_e17_lhloose_mu14
HLT_mu18_mu8noL1

2016(A-D3)
HLT_2e17_lhvloose_nod0
HLT_e17_lhloose_nod0_mu14
HLT_mu20_mu8noL1

2016(D3-)
HLT_2e17_lhvloose_nod0
HLT_e17_lhloose_nod0_mu14
HLT_mu22_mu8noL1
\end{verbatim}
\end{frame}

\begin{frame}{Object Definitions}
\small
Tool: AnalysisBase 2.4.31, SUSYTools-00-08-60\\

\centering
\begin{table}
\small
\begin{tabularx}{\textwidth}{p{1.5cm} | p{3cm} | p{3cm} | p{3cm}}
& \textbf{Electron} & \textbf{Muon} & \textbf{Jet}\\
\hline
\textbf{Baseline}
& - $p_T>10$ GeV \newline - $|\eta^{cluster}| < 2.47$ \newline - LooseAndBLayerLLH
& - $p_T>10$ GeV \newline - $|\eta| < 2.7$ \newline - Medium
& - $p_T>20$ GeV \\
\hline
\textbf{Signal}
& - $p_T > 25$ GeV \newline - $|\eta^{cluster}| < 2.47$ \newline - TightLLH \newline - GradientLoose \newline - $|z_0 \sin \theta| < 0.5$mm \newline - $|d_0/\sigma_{d_0}| < 5$
& - $p_T > 25$ GeV \newline - $|\eta| < 2.7$ \newline - Medium \newline - GradientLoose \newline - $|z_0 \sin \theta| < 0.5$mm \newline - $|d_0/\sigma_{d_0}| < 3$
& - $p_T > 20$ GeV \newline - $|\eta|<2.8$ \newline \newline - $|JVT| > 0.59$ \newline if $p_T < 60$ GeV \newline and $|\eta| < 2.4$
\end{tabularx}
\end{table}

\raggedright
Selection:
\begin{itemize}
%\item Trigger selection
\item Exactly 2 baseline leptons and exactly 2 signal leptons
\end{itemize}

\tiny
Note: \\
Pileup reweighting is applied. \\
Scale factor for reconstruction, isolation, ID and trigger is applied.
\end{frame}

\begin{frame}
\frametitle{averageMu}
\begin{figure}
\includegraphics[width=0.33\textwidth]{averageMu_CR_ISR_OS_ee}
\includegraphics[width=0.33\textwidth]{averageMu_CR_ISR_OS_mumu}
\includegraphics[width=0.33\textwidth]{averageMu_CR_ISR_OS_emu} \\
\caption{Average number of interactions per bunch crossing, for ee channel (left), $\mu\mu$ channel (middle) and e$\mu$ channel (right).}
\end{figure}
\end{frame}

\begin{frame}
\frametitle{Definition of jets}
\normalsize
\begin{itemize}
\item Central jets: $\pt>20$ GeV, $|\eta|<2.4$, no b-tagged
\begin{itemize}
\item This is the run 1 definition.
\end{itemize}
\item B-jets: b-tagged
\end{itemize}
\end{frame}

\begin{frame}
\frametitle{significance calculation}
\begin{itemize}
\item RooStats::NumberCountingUtils::BinomialExpZ(S,B,$\delta$B)
\item $\delta$B = sqrt((0.25)\^{}2 + (nBGError/nBG)\^{}2)
\begin{itemize}
\item use the same definition from Dani.
\end{itemize}
\end{itemize}
\end{frame}

\begin{frame}
\frametitle{definition of variables}
\normalsize
\begin{itemize}
\item HT: Sum of the $p_T$ of all signal jets and the two leptons.
\item R2 = MET / (MET + pt1 + pt2)
\item l12\_dPhi: difference in phi between the two leptons.
\item l12\_MET\_dPhi: difference in phi between MET and the sum of 4-momentum of the two leptons.
\end{itemize}
\end{frame}

\begin{frame}
\small
SUSY scenario:\\
\begin{figure}
\includegraphics[width=0.6\textwidth]{data/photo/Wh.png}
\end{figure}
\end{frame}
\end{document}
